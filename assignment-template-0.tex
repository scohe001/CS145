\documentclass[11pt]{article}

\input{assignment-macros-0}

\setTurnIn{???}
\setAuthor{Stanley Cohen (scohe001)}
\setSID{861114309}
\setHeader

\begin{document}

\begin{problems}

%%%%%%%%%%%%%%%%%%%%%%%%%%%%%%%%%%%%%%%%%%%%%%%%%%%%%%%% 1

\problem

\begin{theorem}[page 12 of MCS]
  If $0\le x \le 2$, then $-x^3 + 4x+1>0$.
\end{theorem}
\begin{longFormProof}
  \begin{block}[blockA]
    {Assume $x$ is an arbitrary real with $0\le x\le 2$.}
    
    \step Then $x$, $2-x$, and $2+x$ are each non-negative.

    \step So, the product of these three terms is also non-negative.  That is,
    \[ x(2-x)(2+x) \ge 0.\]

    \step  Multiplying out the left-hand side gives \( -x^3 + 4x \ge 0 \).

    \step Adding 1 to the both sides gives \( -x^3 + 4x + 1 \ge 1 > 0\).
  \end{block}

  \step By block~\ref{blockA}, if $0\le x\le 2$, then $-x^3+4x+1 > 0$.
\end{longFormProof}

%%%%%%%%%%%%%%%%%%%%%%%%%%%%%%%%%%%%%%%%%%%%%%%%%%%%%%%% 2

\problem

\begin{theorem}[page 13 of MCS]
  If $r$ is irrational, then $\sqrt r$ is also irrational.
\end{theorem}
\begin{longFormProof}
  \begin{block}[blockB]
    {Assume for contradiction 
      that there is an irrational number $r$
      such that $\sqrt r$ is rational.}
    
    \step Since $\sqrt r$ is rational, $\sqrt r = p/q$ for integers $p$ and $q$. 

    \step Squaring both sides, $r = (\sqrt r)^2 = (p/q)^2 = p^2/q^2$.  Hence $r$ is rational.

    \step This contradicts that $r$ is irrational (line~\ref{blockB}).
  \end{block}
  
  \step By block~\ref{blockB},
  there does not exist an irrational number $r$
  such that $\sqrt r$ is rational.

  \step Equivalently, for every irrational number $r$,  $\sqrt r$ is also irrational.
\end{longFormProof}

%%%%%%%%%%%%%%%%%%%%%%%%%%%%%%%%%%%%%%%%%%%%%%%%%%%%%%%% 3

\problem

\begin{theorem}[page 14 of MCS]
  The standard deviation of a sequence of numbers $x_1, x_2,\ldots, x_n$
  is zero if and only if all values are equal to the mean.
\end{theorem}
\begin{longFormProof}
  \begin{block}[blockC]
    {Let $x_1,x_2,\ldots,x_n$ be an arbitrary sequence of numbers}
    
    \step[dev] 
    Let $\mu=\sum_{i=1}^n x_i/n$ be the mean. 
    Let $\sigma=\sqrt{\sum_{i=1}^n (x_i-\mu)^2/n}$ be the standard deviation.

    
    \begin{block}[blockC1]
      {Assume the standard deviation is zero; that is, $\sigma=0$.}
      
      \step From the equation for $\sigma$ in  line~\ref{dev},
      squaring both sides gives
      $\sum_{i=1}^n (x_i-\mu)^2/n = 0$.

      \step Each term in that sum (being a square) is non-negative.

      \step And, since the sum of the terms is zero, each term must be zero:
      $(x_i-\mu)^2/n = 0$ for all $i$.

      \step It follows that $x_i = \mu$ for all $i$.  That is, all values are equal to the mean.
    \end{block}
    
    \step[if] By block~\ref{blockC1},
    if the standard deviation is zero, then all values are equal to the mean.

    \begin{block}[blockC2]
      {Assume all values are equal to the mean; that is, $x_i = \mu$ for all $i$.}

      \step Inspecting line~\ref{dev}, each term in the sum is zero, so $\sigma=0$.
    \end{block}
    
    \step[only if] By block~\ref{blockC2},
    if all values are equal to the mean, then the standard deviation is zero.
    
    \step By lines~\ref{if} and~\ref{only if},
    the standard deviation is zero if and only if each value equals the mean.
  \end{block}
\end{longFormProof}

%%%%%%%%%%%%%%%%%%%%%%%%%%%%%%%%%%%%%%%%%%%%%%%%%%%%%%%% 4

\problem

  \begin{lemma}[from the proof-guide handout]\label{lemma: ave}
    Let $x_1,x_2,\ldots,x_n$ be any finite, non-empty sequence of real numbers.
    If their average equals their maximum
    ($\sum_{i=1}^n x_i/n = \max_{i=1}^n x_i$),
    then they are all equal.
  \end{lemma}

  \smallskip

  Note: the high-level structure of the lemma is of the form
  ``$\forall x\in S.~\text{if } A(x) \text{ then } B(x)$'',
  where $S$ is the set of non-empty sequences of reals,
  $A(x)$ is ``their average in $x$ equals the maximum'',
  and $B(x)$ is ``all numbers in $x$ are equal''.

  \begin{longFormProof}[First long-form proof.]
    \begin{block}[block0]
      {Assume $x_1, x_2,\ldots, x_n$ is an arbitrary finite, non-empty sequence of reals.
      \hfill\emph{(set up ``to show $\forall$'')} }
  
      \step Let $M = \max_i x_i$ be their maximum.

      \begin{block}[block1]
        {Assume the average of the $x_i$'s equals $M$.
        \hfill\emph{(set up ``if then'')}}

        \begin{block}[block2]
          {Assume $j$ is an arbitrary index in $\{1,2,\ldots,n\}$.
          \hfill\emph{(set up ``to show $\forall$'')}}

          \step By definition of the maximum, each $x_i$ is at most $M$, 
          so, by algebra, the average (which equals $M$ by line~\ref{block1}),
          is 
          \[
          M ~=~ \sum_{i=1}^n \frac{x_i}{n} ~\le~ (n-1) \frac{M}{n} + \frac{x_j}{n}. 
          \]

          \step Multiplying the left-hand and right-hand sides by $n$ and simplifying gives
          \(
          M \le x_j
          \).
  
          \step This and (by definition of $M$) $x_j \le M$ yield $x_j = M$.
        \end{block}

        \step By block~\ref{block2}, $\forall j\in\{1,\ldots,n\}.~ x_j = M$.
        So all the numbers are equal.
        \hfill\emph{(by ``to show $\forall$'')}
      \end{block}

      \step By block~\ref{block1}, 
      if the average of the $x_i$'s equals the maximum,
      then all the numbers are equal.
      \\\mbox{}\hfill\emph{(by ``if then'')}
    \end{block}

    \step By block~\ref{block0},
    for any finite, non-empty sequence  $x_1, x_2,\ldots, x_n$ of reals,
    if their average equals their maximum,
    then they are all equal.
    \hfill\emph{(by ``to show $\forall$'')}
  \end{longFormProof}

  \begin{longFormProof}[Alternative long-form proof.]
    \begin{block}[block20]
      {Assume for contradiction that the lemma is false ---
      there exists a non-empty sequence of real numbers, not all equal,
      whose average equals their maximum.
      \hfill\emph{(set up  ``to show not $A$'')}}

      \step[block21] Let $x_1, x_2,\ldots, x_n$ be such a sequence.
      \hfill\emph{(by ``to use $\exists$'')}

      \step Let $M = \max_i x_i$ be the maximum.

      \step By definition of $M$, for each $i$, $x_i \le M$.

      \step By the assumption that they are not all equal,
      $\exists j\in\{1,2,\ldots,n\}.~x_j < M$.

      \step[block22] Fix $j$ such that $x_j < M$.  
      \hfill\emph{(``to use $\exists$'')}

      \step Summing inequality $x_j < M$ and the inequalities $x_i \le M$ 
      (for $i\ne j$) gives $\sum_{i=1}^n x_i < M n$.

      \step Dividing by $n$ gives $\sum_{i=1}^n x_i/n < M$.  
      That is, the average is less than the maximum.

      \step This contradicts the assumption on line~\ref{block21}
      \hfill\emph{(by ``to use not'')}
    \end{block}

    \step By block~\ref{block20}, the lemma is not false.
    \hfill\emph{(by``to show not $A$'')}
  \end{longFormProof}

  %%%%%%%%%%%%%%%%%%%%%%%%%%%%%%%%%%%%%%%%%%%%%%%%%%%%%%%% 5

  \newpage

  \problem

  \begin{theorem}[from the proof-guide handout]
    Let $G=(V,E)$ be any undirected, connected, finite graph,
    where each vertex $v\in V$ has a numeric weight $w(v)$.
    Suppose, for each vertex $v\in V$, the weight of $v$ equals
    the average of the weights of its neighbors.
    Then all vertex weights are the same.
  \end{theorem}

  \smallskip

  Note: the high-level structure of the theorem is of the form
  ``$\forall G\in S.~\text{if } A(G) \text{ then } B(G)$'',
  where $S$ is the set of undirected, connected, finite graphs,
  $A(G)$ is ``each vertex weight in $G$ equals the average of its neighbors weights'',
  and $B(G)$ is ``all vertex weights in $G$ are the same''.
  (Each of the properties $A(G)$ and $B(G)$ are, in turn,
  of the form ``$\forall v\in G.~P(v)$'' for some property $P$.)

  \begin{longFormProof}

    \begin{block}[block30]
      {Assume $G=(V,E)$ and $w$ give an arbitrary vertex-weighted graph as in the theorem.}

      \begin{block}[block31]
        {Assume that the weight of each vertex $v\in V$ 
        equals the average of its neighbors' weights.}

        \step Let $M = \max_{v\in V} w(v)$ be the maximum vertex weight.
        ($M$ is well-defined, as $G$ is finite.)

        \smallskip 
        \hrulefill
        \smallskip 

        \begin{lemma}
          Let $v^*$ be any vertex of maximum weight ($w(v^*)=M$).
          Then all neighbors of $v^*$ also have weight $M$.
        \end{lemma}

        \begin{longFormProof}[Proof of lemma.]
          \step By the choice of $M$, each neighbor's weight is at most $M$. 
          \step By line~\ref{block31} and the choice of $v^*$,
          the average of the neighbors' weights equals $M$.
          \step So, applying Lemma~\ref{lemma: ave} to the neighbors' weights,
          each neighbor's weight must be $M$.
        \end{longFormProof}

        \vspace*{-1em}
        \hrulefill 
        \smallskip

        \begin{block}[block32]
          {Assume $v'$ is an arbitrary vertex in $V$.  
          Let $v^*$ be a vertex of maximum weight, $M$.}

          \step Let $P=(v^*=v_1,v_2,\ldots,v_\ell=v')$ be a path from $v^*$ to $v'$. 
          ($P$ exists, as $G$ is connected.)

          \step By induction on $i=1,2,\ldots,\ell$, and the lemma,
          each vertex $v_i$ on the path has weight $M$.

          \step Hence, the weight of $v'$ equals $M$.
        \end{block}
  
        \step It follows from block~\ref{block32} 
        that every vertex $v\in V$ has weight $M$.
        \hfill\emph{(by ``to show $\forall$'')}
      \end{block} % block31
  
      \step It follows from block~\ref{block31} that,
      if the weight of each vertex $v$ in $G$ equals the average of its neighbors' weights,
      then all vertex weights are equal.
      \hfill\emph{(by ``to show if then'')}
    \end{block} % block30

    \step It follows from block~\ref{block30} that, 
    in any vertex-weighted graph as described in the theorem,
    all vertex weights are equal.
    \hfill\emph{(by ``to show $\forall$'')}

  \end{longFormProof}

  \begin{shortFormProof}
    Let $G$ be as assumed in the theorem, 
    with each vertex weight equal to the average of its neighbors' weights.
    Let $M$ be the maximum vertex weight.
    Consider any vertex $v$ of weight $M$.
    The neighbors of $v$ must also have weight $M$,
    because each has weight at most $M$, and their average is $M$.
    By the same reasoning, the neighbors of the neighbors of $v$
    also have weight $M$, and so on --- inductively, all vertices reachable from $v$
    have weight $M$.
    Since the graph is connected, all vertices are reachable from $v$,
    so all vertices have weight $M$.
  \end{shortFormProof}

  %%%%%%%%%%%%%%%%%%%%%%%%%%%%%%%%%%%%%%%%%%%%%%%%%%%%%%%% 6

  \newpage

  \problem

  Assume that, given any two people, either they have met or not. 
  If every pair in a group has met, call the group a \emph{club}. 
  If every pair in a group has not met, call it a group of \emph{strangers}.

  \begin{theorem}[from MCS page 15 and the proof-guide handout]
    Every group of 6 people includes a club of 3 or a group of 3 strangers.
  \end{theorem}

  \begin{longFormProof}
    \begin{block}[block40]
      {Assume $C$ is an arbitrary collection of six people.
        Let $x$ denote any one of the six people.

        Say that $C$ is \emph{good} if $C$ includes a club of 3 or a group of 3 strangers.
        We prove that $C$ is good.}

      \begin{case}[case1]
        {Assume that, among the five people other than $x$, at least three have met $x$.}

        \begin{case}[case11]
          {Suppose that no two among those three have met.}

          \step Those three people are a group of strangers, 
          so $C$ is good.
        \end{case}

        \begin{case}[case12]
          {Suppose that some pair among those three have met.}

          \step That pair, and $x$, have all met each other, so $C$ has a club of three, 
          so $C$ is good.
        \end{case}

        \step Within case~\ref{case1}, case~\ref{case11} or~\ref{case12} must hold;
        $C$ is good in each,
        so $C$ is good in case~\ref{case1}.
      \end{case}
      
      \begin{case}[case2]
        {Assume that, among the five people other than $x$, 
        at least three have not met $x$.}

        \begin{case}[case21]
          {Suppose that every pair among those three have met.}
      
          \step Those three are a club, so $C$ is good.
        \end{case}

        \begin{case}[case22]
          {Suppose that some pair among those three have not met.}

          \step That pair, and $x$, have all not met, so form a group of three strangers, 
          so $C$ is good.
        \end{case}

        \step  Within case~\ref{case2}, case~\ref{case21} or~\ref{case22} must hold;
        $C$ is good in each, 
        so $C$ is good in case~\ref{case2}.
      
      \end{case}

      \step Case~\ref{case1} or~\ref{case2} must hold (verify!);
      $C$ is good in each,
      so $C$ is good.
    \end{block}

    \step By block~\ref{block40}, every group of 6 people is good,
    that is, includes a club of 3 or a group of 3 strangers.
  \end{longFormProof}
  
  \begin{shortFormProof}
    Fix any one of the six people.  Call that person $x$.
    \begin{description}
    \item[Case 1.] Suppose that, among the five people other than $x$, 
      at least three have met $x$.

      If no two among those other three have met, 
      then those three are a group of three strangers,  so we are done.
      Otherwise, some pair among those three have met.
      That pair, together with $x$, form a club of three, so we are done.

    \item[Case 2.] Otherwise, among the five people other than $x$, 
      at least three have not met $x$.
      The reasoning in this case is symmetric:
      If no pair among those three have met, then those three are a club,
      so we are done.
      Otherwise, some pair among those three have not met.
      That pair, together with $x$, are a group of three strangers.
    \end{description}
  \end{shortFormProof}
\end{problems}
\end{document}

%%% Local Variables:
%%% mode: latex
%%% TeX-master: t
%%% End:
