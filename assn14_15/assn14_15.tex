\documentclass[10pt]{article}

\input{assignment-macros-2}

% TODO: CHANGE NAME AND SID TO YOURS:

\setTurnIn{14-15}  

\setAuthor{Stanley Cohen (scohe001)}
\setSID{861114309}
\setHeader

\begin{document}

\emph{The problems for this assignment will use the following graph, $G$}

\includegraphics[scale=0.5]{graph}

\begin{problems}

  \problem

  Here is an optimal solution vector, $X$ for the primal LP for shortest paths from $A$ to $E$ in $G$:

  $(x_{AB}, x_{AC}, x_{BC}, x_{CB}, x_{BD}, x_{BE}, x_{CD}, x_{CE}, x_{ED}) = (0,1,0,1,0,1,0,0,0)$ Yielding an optimal value of 6.

  \medskip 
  \problem
  Here is an optimal solution vector, $\pi$, for the dual LP for shortest paths from $A$ to $E$ in $G$:

  $(\pi_A, \pi_B, \pi_C, \pi_D, \pi_E) = (0, 3, 2, 5, 6)$ Yielding an optimal value of 6.
  
  \medskip 
  \problem

\begin{lemma}
  The dual solution $\pi$ above is optimal.
\end{lemma}

\begin{longFormProof}

    \step By problem 1, we know that 6 is a feasible solution to the primal LP.

    \step Therefore, by weak duality, every feasible solution to the dual has value at most 6.

    \step Since we've found a feasible solution, $\pi$ to the dual equal to the upper bound, $\pi$ must be optimal.
\end{longFormProof}

\end{problems}

\end{document}

%%% Local Variables:
%%% mode: latex
%%% TeX-master: t
%%% End:
