\documentclass[11pt]{article}

\input{assignment-macros-0}
\newcommand\tab[1][1cm]{\hspace*{#1}}

% TODO: CHANGE NAME AND SID TO YOURS:

\setTurnIn{3}
\setAuthor{Stanley Cohen (scohe001)}
\setSID{861114309}
\setHeader

\begin{document}

% TODO: DELETE THIS LINE:
Download this LaTeX file (along with \textsf{assignment-macros-0.tex}) to use as a template for turn-in 3.

% SEE assignment-template-0.tex FOR LATEX PROOF EXAMPLES.

\begin{enumerate}

  %%%%%%%%%%%%%%%%%%%%%%%%%%%%%%%%%%%%%%%%%%%%%%%%%%%%%%%% 1

  \item % 1. 

  \begin{enumerate}
  \item % 1(a) 

    \begin{enumerate}
    \item

      My partner in this assignment was Michael (mpena010). His chosen problem was At least one round, at least one square. The problem states that you have a set of items where each item has a color trait and a shape trait. Each is red, blue, round or square. There is at least one red, one blue, one round and one square object. The problem asks us to prove that there must exist at least two objects that are different both in color and in shape. Michael split this problem into the following 6 cases:
      \begin{itemize}
        \item The case where every item but one is blue. In this case, since we must have a square and a circular object, regardless of what shape the blue object is, there must be a complimentary red object.
        \item The same case as the first, but swapping red for blue.
        \item The same case as the first, but dealing with all round and one square.
        \item The same case as the first, but dealing with all square and one round.
        \item The case in which we've examined every item but one and they've all been the same color and shape. By the definition of the problem, the last item must be the other and color and other shape.
        \item The last case he explained is the case when we've "already found" a pair of different color and shape in which we're already done.
      \end{itemize}

    \item

      While I understand Michael's thought process, I thought his wording was not very good. The way he chose his cases for the exhaustive proof didn't feel complete or standard to say the least.

    \item

      After thinking more about it, I'm still not sure if this exhaustive proof covers every case. When I asked him about the case where you have 5 blues, 5 reds, 5 square and 5 circles, he pointed to his last case, saying it was the standard case. However, I don't think he ever really proved that case. I'm still not entirely sure as the wording and the way he went about the proof confused me.

    \end{enumerate}
    
  \item % 1(b)

    %%%%%%%%%%%%%%%%%%%%%%%%%%%%%%%%%%%%%%%%%%%%%%%%%%%%%%% 1(b)
     
    \begin{theorem}[Problem 1]
     If $x = \log_9(12)$, then $x$ is irrational.
    \end{theorem}
    \begin{longFormProof}
     \step By the definition of logs, we can say $x = \log_9(4*3) = \log_9(3) +
             \log_9(4) = \frac{1}{2} + \log_9(4)$
     
     \smallskip
     \hrulefill
     \smallskip
     
           \begin{lemma}
             Any rational plus an irrational will be an irrational number.
           \end{lemma}
     
           \begin{longFormProof}[Proof of lemma.]
             \begin{block}[blockL]
                {Assume for contradiction rational + irrational = rational}
             \step We can write the first rational as $a/b$ and the second as $c/d$
                for some $a, b, c, d \in \mathbb{Z}$ so $a/b +$ irrational $=c/d$.
             \step Subtracting $a/b$ from each side yields irrational $=c/d - a/b$
             \step $c/d - a/b$ can be represented as the fraction $\frac{ad-bc}{bd}$, which is rational, contradicting a.
             \end{block}
             \step Therefore, a rational + an irrational = an irrational.
           \end{longFormProof}
     
     \vspace*{-1em}
     \hrulefill
     \smallskip
     
     \step For $x$ to be rational, $y = \log_9(4)$ must be rational by Lemma 1.
     \step $y = \log_9(4)$ can be rewritten as $9^y=4$.
     \step Taking the square root of each side yields $3^y=2$.
     \begin{block}[blockA]
       {Assume for contradiction $y$ is rational, $y=p/q$ for some $p, q \in \mathbb{Z}$}
       
       \step we can rewrite $9^y=4$ as $3^{p/q}=2$ or $3^p=2^q$.
     
       \step $3^p$ will always be odd, while $2^p$ will always be even.
            Therefore, there can't exist any $p, q \in \mathbb{Z}$ to satisfy this equation.
     
       \step By step 3.5, $p$ and $q$ cannot be integers. Therefore y is not rational by contradiction.
     \end{block}
     
     \step By block~\ref{blockA}, and Lemma 1, $x$ cannot be rational
        since it is made up of the sum of an irrational number, y.
    \end{longFormProof}

  \item % 1(c)

    In version 2.0, I've added a Lemma in an attempt to go deeper into my explanation. I fixed the wording that was talked about over iLearn. I also generally rewrote the entire proof with logic rules in mind in an attempt to make it more easily readable.\\
    \tab My partner in class really didn't give me too much feedback. He told me he liked my proof and was sure it was complete.

  \end{enumerate}

  \item % 2

  \begin{enumerate}
  \item % 2(a) 

    \begin{enumerate}
    \item

      I was skeptic for Kristen and then Sehoon.\\
      \tab Kristen chose the Elizabeth hides from John problem. She talked about how we
        have 3 possible ways for John to take from Elizabeth, looking at the parity of
        $card-coin$ in each:
      \begin{itemize}
      \item -1 Coin, -1 Card, +1 Candy ($card-coin$ stays odd)
      \item -1 Coin, +1 Card, -1 Candy ($card-coin$ stays odd)
      \item +1 Coin, -1 Card, -1 Candy ($card-coin$ stays odd)
      \end{itemize}
      \tab Since $card-coin$ stays odd regardless of the move made by John, it's impossible for
      them to be equal meaning Elizabeth can't be left with only Candies.\\
      \\
      \tab Sehoon chose the At least one round, at least one square problem. He chose
      to do an exhaustive proof. He said that by the definition of the problem, there must
      be some red item, $x$ and some blue item, $y$.\\
      \tab Sehoon then iterated through each possible combination of their shapes:\\
      \tab \textbf{\textit{if}} $x$ is round\\
      \tab \tab \textbf{\textit{if}} $y$ is square then we've satisfied the problem.\\
      \tab \tab \textbf{\textit{else if}} $y$ is round then there must be a square 
                                          object somewhere that's either red or blue 
                                          and either way we've satisfied the problem.\\
      \\
      \tab \textbf{\textit{if}} $x$ is square\\
      \tab \tab \textbf{\textit{if}} $y$ is round then we've satisfied the problem.\\
      \tab \tab \textbf{\textit{else if}} $y$ is square then there must be a round 
                                          object somewhere that's either red or blue 
                                          and either way we've satisfied the problem.\\

    \item

      I told Kristen that I liked her proof and believed it to be complete. 
      I told her that it looked like she was attempting to prove an \textbf{or}, 
      so she should use three assumptions to make a three way or.\\

      \tab I really enjoyed Sehoon's proof. In comparison to the other At least 
      one round, at least one square solution I saw in the last skeptic exercise, 
      his looked extremely refined and was a pleasure to hear. I wasn't sure 
      what to tell him as far as wording it better for the proof, but his logic 
      and way of thinking were definitely sound. 

    \item

      Not much else to say after thinking about it. The two of them had very 
      clean thorough proofs that I still agree are easily complete.

    \end{enumerate}
    
  \item % 2(b)

      \begin{theorem}[Problem 7]
       (Handshakes after the party) $N + 1$ couples attend a dinner. At the end of the dinner, 
        one person (Alice) asks each other person to write down the number of distinct people 
        with whom he or she shook hands at the dinner. Surprisingly, all numbers, 
        $0, 1, 2, ..., 2N$ are written down. Assuming that no person shook hands with their 
        partner, how many people did Alice shake hands with at the dinner?
      \end{theorem}

      \begin{longFormProof}
       \step Assume we have a party of $N + 1$ couples where indiduals have shaken 
              hands with others but not with their own partner.
       \step Call the individuals of one such couple Alice and Bob.
       \step Assume that besides Alice, the number of hands shaken by each individual 
            ($h(i)$ for some individual $i$) is a unique integer in the range $[0,2N]$.
       \step By 3, there must exist an individual who's shaken hands with $2N$ other individuals.
       \begin{block}[blockA]
          {Assume for contradiction that Bob is the person who's shaken hands with $2N$ individuals.}

          \step Besides Alice and Bob, there are $2N$ guests.
          \step By 1, Bob cannot shake hands with Alice.
          \step By 5.1 and 5.2, Bob must've shaken hands with everyone but Alice.
          \step Handshaking is associative, so everyone but Alice has shaken hands with Bob.
          \step Contradiction by 3, someone (besides Alice) must've shaken hands with no one, but everyone shook hands with Bob.
       
       \end{block}

       \step By block~\ref{blockA}, Bob did not shake hands with $2N$ people.
       
       \smallskip
       \hrulefill
       \smallskip
       
             \begin{lemma}
               Let $(x,y)$ be a couple neither of which are Alice, with $h(x)=2N$.  Then $h(y)=0$.
             \end{lemma}
       
             \begin{longFormProof}[Proof of lemma.]
               \begin{block}[blockL]
                  {Assume for contradiction $h(y) \neq 0$.}

                  \step since $h(x) = 2N$, $x$ cannot shake hands with y and there are $2N$ 
                        people other than $x$ and $y$ at the party, we know x must’ve shaken 
                        hands with everyone but $y$.
                  \step Since hand shaking is associative, everyone but $y$ has shaken hands with $x$.
                  \step $h(y) \neq 0$ by a, but no one else has shaken hands with 0 people 
                        either, which is a contradiction by 3.
               \end{block}
                \step Therefore, if $h(x) = 2N$, it implies $h(y) = 0$.
             \end{longFormProof}
       
       \vspace*{-1em}
       \hrulefill
       \smallskip
       
       \step By 3 and Lemma 1, there is some couple $(x,y)$ where $h(x)=2N, h(y)=0$.

      \smallskip
       \hrulefill
       \smallskip
       
             \begin{lemma}
               Removing $(x,y)$ from the party results in a new party that still adheres to the rules of 1-3 where Alice has shaken hands with 1 less person.
             \end{lemma}
       
             \begin{longFormProof}[Proof of lemma.]
                \step If we ignore the couple $(x,y)$, we're left with a party whose 
                      handshakes now range in $[1,2N-1]$.
                \step Besides $x$ and $y$, there are $2N$ other guests, meaning $x$ 
                      must've shaken hands with each of them, while $y$ has shaken 
                      hands with none of them.
                \step Removing $x$ and $y$ from the party then reduces everyone else's handshake 
                      number by 1, resulting in handshakes in the range $[0, 2N-2]$ or $[0, 2(N-1)]$.
                \step Call $n = N-1$.
                \step We now have a party of $n+1$ individuals, where still no individuals have 
                      shaken hands with their own partners. Besides Alice, the number of hands 
                      shaken by each individual is a unique integer in the range $[0,2n]$.
                \step Alice's handshsakes have decreased 1 by c, and this new party still 
                      adheres to 1-3 by e.
             \end{longFormProof}
       
       \vspace*{-1em}
       \hrulefill
       \smallskip

       \step Because we've started with a party that's valid by 1-3 and any removal of the couple $(a,b)$ where $h(a)=2N, h(b)=0$ will still result in a valid party, we can continue to remove this couple and create a new party, forcing Alice to shake one less hand each time.
       \step We can repeat this process $N$ times, until Alice and Bob's couple is the only one left.
       \step Since we can go through $N$ removals and by Lemma 2, a removal results in one less handshake for Alice, Alice must've shaken hands with $N$ people in the initial party.



      \end{longFormProof}

  \item % 2(c)
    I left the first half of the proof the same as I felt it was fairly solid. I 
    updated the wording of Lemma 1 in accordance with the iLearn comment. I also 
    completely rewrote the back half of the proof in an attempt to make it more 
    understandable. I established Lemma 2 and broke the whole process into more 
    steps. In all honesty, it still feels like I'm lacking quite the right language 
    to explain my thoughts entirely, but I'm happy with this.\\
    \tab Because of a mixup with the homework, I brought the wrong assignment to 
    class and didn't get skeptic feedback on this one.

  \end{enumerate}

\end{enumerate}
\end{document}

%%% Local Variables:
%%% mode: latex
%%% TeX-master: t
%%% End:
