\documentclass[11pt]{article}

\input{assignment-macros-2}

\setTurnIn{extra 5}
\setAuthor{Stanley Cohen (scohe001)}
\setSID{861114309}
\setHeader

\begin{document}

\begin{problems}

  %%%%%%%%%%%%%%%%%%%%%%%%%%%%%%%%%%%%%%%%%%%%%%%%%%%%%%%% 5

  \newpage

  \problem

  \begin{theorem}[All pairs have different sums]
      Prove or disprove: It is possible to label each square of an 8 x 8 checkerboard with an integer, 
      so that no two squares are labeled with the same integer and no two adjacent squares 
      (up, down, left, right) have labels that differ by 8 or more.
  \end{theorem}

  \smallskip

  \begin{longFormProof}

    \begin{block}[A]
      {Assume you have filled some chessboard with integers to satisfy the theorem.}

      \step Construct a set $S$ of chessboard squares by doing the following...

      \step Add the square with the smallest integer not in $S$ into $S$, stopping when the squares in $S$ are touching at least $8$ squares not in $S$.

      \step Call $T$ the set of chessboard squares adjacent to those in $S$ but not in $S$.

      \step Call the largest numbered square in $S$, $s$, where the number in $s$ is $num(s)$.

      \step Since there are $8$ squares in $T$ and the numbers in the squares in $T$ must be unique integers $> num(s)$, the lower bound on the largest numbered square in $T$ will be $num(s) + 8$.

      \step However, by the definition of the theorem, there is no square in $S$ that the largest square in $T$ can be adjacent to.

      \step Contradiction.

    \end{block}
    
    \step By block~\ref{A} you cannot fill a chessboard with integers to satisfy the theorem.

    \step That is, the theorem is false.
    
  \end{longFormProof}


\end{problems}
\end{document}

%%% Local Variables:
%%% mode: latex
%%% TeX-master: t
%%% End:
