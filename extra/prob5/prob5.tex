\documentclass[11pt]{article}

\usepackage{xifthen}% \isempty
\usepackage{stringstrings}% \substring

\usepackage{nameref}
\makeatletter
\newcommand*{\currentlabel}{\@currentlabel}
\makeatother

\usepackage{amsmath, amsthm, amssymb}
\usepackage{fullpage}
\usepackage[colorlinks=false,urlcolor=blue,pageanchor=true]{hyperref}
\usepackage{xspace}
\usepackage{graphicx}
\usepackage[normalem]{ulem}
\usepackage{enumitem}
\usepackage{adjustbox}

\newtheorem{lemma}{Lemma}
\newtheorem{definition}{Definition}
\newtheorem{theorem}{Theorem}
\newtheorem{corollary}{Corollary}
\newtheorem{claim}{Claim}

\newcommand{\E}{\operatorname{E}}
\newcommand{\giv}{\,|\,}
\newcommand{\Z}{\mathbb{Z}}
\newcommand{\Zp}{\mathbb{Z}_{\ge 0}}
\newcommand{\R}{\mathbb{R}}
\newcommand{\Rp}{\mathbb{R}_{\ge 0}}
\newcommand{\N}{\mathbb{N}}
\newcommand{\Np}{\mathbb{N}_{\ge 0}}

\newcommand{\OP}[1]{{\small\sc #1}\xspace}
\newcommand{\Insert}{\OP{Insert}}
\newcommand{\FindMin}{\OP{Find-Min}}
\newcommand{\DeleteMin}{\OP{Delete-Min}}
\newcommand{\DecreaseKey}{\OP{Decrease}}
\newcommand{\Merge}{\OP{Merge}}
\newcommand{\Cut}{\OP{Cut}}

\setlength{\parskip}{2pt}
% \setlength{\parindent}{0in}

\setlength\fboxsep{0pt}
\newlist{steps}{enumerate}{9}
\setlist[steps]{wide,label*=\arabic*.,topsep=1pt,parsep=1pt,partopsep=1pt,itemsep=0pt, labelindent=2pt}%
\setlist[steps,1]{ref=\arabic*}
\setlist[steps,2]{ref=\arabic{stepsi}.\arabic*}
\setlist[steps,3]{ref=\arabic{stepsi}.\arabic{stepsii}.\arabic*}
\setlist[steps,4]{ref=\arabic{stepsi}.\arabic{stepsii}.\arabic{stepsiii}.\arabic*}
\setlist[steps,5]{ref=\arabic{stepsi}.\arabic{stepsii}.\arabic{stepsiii}.\arabic{stepsiv}.\arabic*}
\setlist[steps,6]{ref=\arabic{stepsi}.\arabic{stepsii}.\arabic{stepsiii}.\arabic{stepsiv}.\arabic{\stepsv}.\arabic*}
\setlist[steps,7]{ref=\arabic{stepsi}.\arabic{stepsii}.\arabic{stepsiii}.\arabic{stepsiv}. \arabic{\stepsv}.\arabic{\stepsvi}.\arabic*}
\setlist[steps,8]{ref=\arabic{stepsi}.\arabic{stepsii}.\arabic{stepsiii}.\arabic{stepsiv}. \arabic{\stepsv}.\arabic{\stepsvi}.\arabic{\stepsvii}.\arabic*}
\setlist[steps,9]{ref=\arabic{stepsi}.\arabic{stepsii}.\arabic{stepsiii}.\arabic{stepsiv}. \arabic{\stepsv}.\arabic{\stepsvi}. \arabic{\stepsvii}.\arabic{\stepsviii}.\arabic*}

\newlist{asteps}{enumerate}{6}
\setlist[asteps]{wide,topsep=1pt,parsep=1pt,partopsep=1pt,itemsep=0pt, labelindent=1em}%
\setlist[asteps,1]{
  label=\alph*.,
  ref=\alph*
}
\setlist[asteps,2]{
  label=\alph{astepsi}.\arabic*.,
  ref=\alph{astepsi}.\arabic*
}
\setlist[asteps,3]{
  label=\alph{astepsi}.\arabic{astepsii}.\arabic*.,ref=\alph{astepsi}.\arabic{astepsii}.\arabic*
}
\setlist[asteps,4]{
  label=\alph{astepsi}.\arabic{astepsii}.\arabic{astepsiii}.\arabic*.,
  ref=\alph{astepsi}.\arabic{astepsii}.\arabic{astepsiii}.\arabic*
}
\setlist[asteps,5]{
  label=\alph{astepsi}.\arabic{astepsii}.\arabic{astepsiii}.\arabic{astepsiv}.\arabic*.,
  ref=\alph{astepsi}.\arabic{astepsii}.\arabic{astepsiii}.\arabic{astepsiv}.\arabic*
}
\setlist[asteps,6]{
  label=\alph{astepsi}.\arabic{astepsii}.\arabic{astepsiii}.\arabic{astepsiv}.\arabic{\astepsv}.\arabic*.,
  ref=\alph{astepsi}.\arabic{astepsii}.\arabic{astepsiii}.\arabic{astepsiv}.\arabic{\astepsv}.\arabic*
}

\newlist{problems}{enumerate}{9}
\setlist[problems]{wide, label=\textbf{Problem \arabic*},topsep=1pt,parsep=1pt,partopsep=1pt,itemsep=0pt, labelindent=0pt}%

\newcommand{\STEPS}{steps}
\newcommand{\SHIFT}{\renewcommand{\SHIFT}{\renewcommand{\STEPS}{asteps}}}

\newcommand{\maybeLabel}[1]{\ifthenelse{\isempty{#1}}{}{\label{#1}}}
\newcommand{\maybeItem}[1]{\ifthenelse{\isempty{#1}}{\item}{\item[#1]}}

\newcommand{\problem}[1][]{\maybeItem{#1}~\par}

\newenvironment{longFormProof}[1][Proof (long form).]
{\SHIFT\begin{proof}[#1]~\par\begin{\STEPS}}{\end{\STEPS}\vspace*{-1ex}\end{proof}}

\newenvironment{shortFormProof}[1][Proof (short form).]
{\SHIFT\begin{proof}[#1]}{\end{proof}}

% \makeatletter
% \newcommand{\caseitem}{\@sitem}
% \newcommand{\@sitem}{%
%   \refstepcounter{\@enumctr}%
%   \item[\textbullet~\csname label\@enumctr\endcsname]}
% \makeatother

\newcommand{\step}[1][]{\item\maybeLabel{#1}}

\newcommand{\comment}[1]{\hfill{\footnotesize\emph{#1}}}

\newcommand{\lineacross}{\par\vspace*{-0.7\baselineskip}\noindent\hrulefill\par}

\newenvironment{block}[2][]
{\step[#1]  #2 \begin{\STEPS}}{\end{\STEPS}}

\newenvironment{case}[2][]
{\step[#1] (case \currentlabel) \emph{#2}\begin{\STEPS}}{\end{\STEPS}}

\newcommand{\setHeader}{\markboth
{\footnotesize CS 145 turn-in \turnIn, \today, by \author, \SID}
{\footnotesize CS 145 turn-in \turnIn, \today, by \author, \SID}}

\newcommand{\setTurnIn}[1]{\newcommand{\turnIn}{#1}}
\newcommand{\setAuthor}[1]{\renewcommand{\author}{#1}}
\newcommand{\setSID}[1]{\newcommand{\SID}{#1}}

\pagestyle{myheadings}
\addtolength{\headsep}{0.3in}
\addtolength{\topmargin}{-0.4in}
\addtolength{\textheight}{0.4in}

%%% Local Variables:
%%% mode: latex
%%% TeX-master: "assignment-template"
%%% End:


\setTurnIn{extra 5}
\setAuthor{Stanley Cohen (scohe001)}
\setSID{861114309}
\setHeader

\begin{document}

\begin{problems}

  %%%%%%%%%%%%%%%%%%%%%%%%%%%%%%%%%%%%%%%%%%%%%%%%%%%%%%%% 5

  \newpage

  \problem

  \begin{theorem}[All pairs have different sums]
      Prove or disprove: It is possible to label each square of an 8 x 8 checkerboard with an integer, 
      so that no two squares are labeled with the same integer and no two adjacent squares 
      (up, down, left, right) have labels that differ by 8 or more.
  \end{theorem}

  \smallskip

  \begin{longFormProof}

    \begin{block}[A]
      {Assume you have filled some chessboard with integers to satisfy the theorem.}

      \step Construct a set $S$ by doing the following...

      \step Add the square with the smallest integer not in $S$ into $S$, 
            stopping when either every row has at least one square in $S$ or when every column has at least one square in $S$.

      \step Without loss of generality, we can assume that every row will have at least one square in $S$ (if it's columns, just rotate the chessboard).

      \step Call $T$ the set of chessboard squares adjacent to those in $S$ but not in $S$.

      \step Consider the largest numbered square in $S$, $s$, where the number in $s$ is $num(s)$.

      \step Before adding $s$ to $S$, there were no squares in $S$ in this row.

      \step Likewise, no row was full (if one had been, we would have already stopped adding to $S$ since there was a square in every column).

      \step Therefore, after the addition of $s$ into $S$, no row will be completely contained in $S$.

      \step Conversely, every row will have at least one square in $S$.

      \step Therefore, there must be at least one square in $T$ in every row.

      \step That is, $|T| \geq 8$.



      \step Since there are at least $8$ squares in $T$ and the numbers in the squares in $T$ must be unique integers $> num(s)$, the lower bound on the largest numbered square in $T$ will be $num(s) + 8$.

      \step However, by the definition of the theorem, there is no square in $S$ that the largest square in $T$ can be adjacent to.

      \step Contradiction.

    \end{block}
    
    \step By block~\ref{A} you cannot fill a chessboard with integers to satisfy the theorem.

    \step That is, the theorem is false.
    
  \end{longFormProof}


\end{problems}
\end{document}

%%% Local Variables:
%%% mode: latex
%%% TeX-master: t
%%% End:
