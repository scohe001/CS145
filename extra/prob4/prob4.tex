\documentclass[11pt]{article}

\input{assignment-macros-2}

\setTurnIn{extra 4}
\setAuthor{Stanley Cohen (scohe001)}
\setSID{861114309}
\setHeader

\begin{document}

\begin{problems}

  %%%%%%%%%%%%%%%%%%%%%%%%%%%%%%%%%%%%%%%%%%%%%%%%%%%%%%%% 5

  \newpage

  \problem

  \begin{theorem}[Tower of Hanoi]
      You are given a subset $S$ of $\{1,2,...,100\}$ with the following property:
        for every quadruple $u,v,w,x$ of distinct numbers in $S$, the sum of $u$ and $v$ 
        differs from the sum of $w$ and $x$.  Must the size of $S$ be at most fifteen?
        Prove your answer.
  \end{theorem}

  \smallskip

  \begin{longFormProof}

    \begin{block}[A]
      {Assume we have some valid set, $S$ of size 15.}

      \step By definition, for any quadruple of distinct numbers $u,v,w,x \in S$, $u + v \neq w + x$.

      \step In other words, $u - w \neq x - v$.

      \step That is, given any two pairs, the differences between the pairs must differ.

      \step Consider solely pairs made of consecutively numbers in $S$ (when $S$ is sorted in ascending order).

      \step We can make 14 pairs of these consecutive numbers, each of which must have a unique difference.

      \step At minimum, The differences between the pairs will be in the set $\{1, 2,..., 14\}$.

      \step However, $\sum_{n=1}^{14} n = 105$, which means that the largest number in $S$ has a lower bound of 106 (since we started at 1 not 0).

      \step This is a contradiciton since every number in $S$ must be in the set $\{1,2,...,100\}$.

    \end{block}

    \step By block~\ref{A}, the size of $S$ cannot be 15.
    
  \end{longFormProof}


\end{problems}
\end{document}

%%% Local Variables:
%%% mode: latex
%%% TeX-master: t
%%% End:
