\documentclass[11pt]{article}

\usepackage{xifthen}% \isempty
\usepackage{stringstrings}% \substring

\usepackage{nameref}
\makeatletter
\newcommand*{\currentlabel}{\@currentlabel}
\makeatother

\usepackage{amsmath, amsthm, amssymb}
\usepackage{fullpage}
\usepackage[colorlinks=false,urlcolor=blue,pageanchor=true]{hyperref}
\usepackage{xspace}
\usepackage{graphicx}
\usepackage[normalem]{ulem}
\usepackage{enumitem}

\newtheorem{lemma}{Lemma}
\newtheorem{definition}{Definition}
\newtheorem{theorem}{Theorem}
\newtheorem{corollary}{Corollary}
\newtheorem{claim}{Claim}

\newcommand{\E}{\operatorname{E}}
\newcommand{\giv}{\,|\,}
\newcommand{\Z}{\mathbb{Z}}
\newcommand{\Zp}{\mathbb{Z}_{\ge 0}}
\newcommand{\R}{\mathbb{R}}
\newcommand{\Rp}{\mathbb{R}_{\ge 0}}
\newcommand{\N}{\mathbb{N}}
\newcommand{\Np}{\mathbb{N}_{\ge 0}}

\newcommand{\OP}[1]{{\small\sc #1}\xspace}
\newcommand{\Insert}{\OP{Insert}}
\newcommand{\FindMin}{\OP{Find-Min}}
\newcommand{\DeleteMin}{\OP{Delete-Min}}
\newcommand{\DecreaseKey}{\OP{Decrease}}
\newcommand{\Merge}{\OP{Merge}}
\newcommand{\Cut}{\OP{Cut}}

\setlength{\parskip}{2pt}
% \setlength{\parindent}{0in}

\setlength\fboxsep{0pt}
\newlist{steps}{enumerate}{9}
\setlist[steps]{wide,label*=\arabic*.,topsep=1pt,parsep=1pt,partopsep=1pt,itemsep=0pt, labelindent=2pt}%
\setlist[steps,1]{ref=\arabic*}
\setlist[steps,2]{ref=\arabic{stepsi}.\arabic*}
\setlist[steps,3]{ref=\arabic{stepsi}.\arabic{stepsii}.\arabic*}
\setlist[steps,4]{ref=\arabic{stepsi}.\arabic{stepsii}.\arabic{stepsiii}.\arabic*}
\setlist[steps,5]{ref=\arabic{stepsi}.\arabic{stepsii}.\arabic{stepsiii}.\arabic{stepsiv}.\arabic*}
\setlist[steps,6]{ref=\arabic{stepsi}.\arabic{stepsii}.\arabic{stepsiii}.\arabic{stepsiv}.\arabic{\stepsv}.\arabic*}
\setlist[steps,7]{ref=\arabic{stepsi}.\arabic{stepsii}.\arabic{stepsiii}.\arabic{stepsiv}. \arabic{\stepsv}.\arabic{\stepsvi}.\arabic*}
\setlist[steps,8]{ref=\arabic{stepsi}.\arabic{stepsii}.\arabic{stepsiii}.\arabic{stepsiv}. \arabic{\stepsv}.\arabic{\stepsvi}.\arabic{\stepsvii}.\arabic*}
\setlist[steps,9]{ref=\arabic{stepsi}.\arabic{stepsii}.\arabic{stepsiii}.\arabic{stepsiv}. \arabic{\stepsv}.\arabic{\stepsvi}. \arabic{\stepsvii}.\arabic{\stepsviii}.\arabic*}

\newlist{asteps}{enumerate}{6}
\setlist[asteps]{wide,topsep=1pt,parsep=1pt,partopsep=1pt,itemsep=0pt, labelindent=1em}%
\setlist[asteps,1]{
  label=\alph*.,
  ref=\alph*
}
\setlist[asteps,2]{
  label=\alph{astepsi}.\arabic*.,
  ref=\alph{astepsi}.\arabic*
}
\setlist[asteps,3]{
  label=\alph{astepsi}.\arabic{astepsii}.\arabic*.,ref=\alph{astepsi}.\arabic{astepsii}.\arabic*
}
\setlist[asteps,4]{
  label=\alph{astepsi}.\arabic{astepsii}.\arabic{astepsiii}.\arabic*.,
  ref=\alph{astepsi}.\arabic{astepsii}.\arabic{astepsiii}.\arabic*
}
\setlist[asteps,5]{
  label=\alph{astepsi}.\arabic{astepsii}.\arabic{astepsiii}.\arabic{astepsiv}.\arabic*.,
  ref=\alph{astepsi}.\arabic{astepsii}.\arabic{astepsiii}.\arabic{astepsiv}.\arabic*
}
\setlist[asteps,6]{
  label=\alph{astepsi}.\arabic{astepsii}.\arabic{astepsiii}.\arabic{astepsiv}.\arabic{\astepsv}.\arabic*.,
  ref=\alph{astepsi}.\arabic{astepsii}.\arabic{astepsiii}.\arabic{astepsiv}.\arabic{\astepsv}.\arabic*
}

\newlist{problems}{enumerate}{9}
\setlist[problems]{wide, label=\textbf{Problem \arabic*},topsep=1pt,parsep=1pt,partopsep=1pt,itemsep=0pt, labelindent=0pt}%

\newcommand{\STEPS}{steps}
\newcommand{\SHIFT}{\renewcommand{\SHIFT}{\renewcommand{\STEPS}{asteps}}}

\newcommand{\maybeLabel}[1]{\ifthenelse{\isempty{#1}}{}{\label{#1}}}
\newcommand{\maybeItem}[1]{\ifthenelse{\isempty{#1}}{\item}{\item[#1]}}

\newcommand{\problem}[1][]{\maybeItem{#1}~\par}

\newenvironment{longFormProof}[1][Proof (long form).]
{\SHIFT\begin{proof}[#1]~\par\begin{\STEPS}}{\end{\STEPS}\vspace*{-1ex}\end{proof}}

\newenvironment{shortFormProof}[1][Proof (short form).]
{\SHIFT\begin{proof}[#1]}{\end{proof}}

% \makeatletter
% \newcommand{\caseitem}{\@sitem}
% \newcommand{\@sitem}{%
%   \refstepcounter{\@enumctr}%
%   \item[\textbullet~\csname label\@enumctr\endcsname]}
% \makeatother

\newcommand{\step}[1][]{\item\maybeLabel{#1}}

\newcommand{\comment}[1]{\hfill{\footnotesize\emph{#1}}}

\newcommand{\lineacross}{\par\vspace*{-0.7\baselineskip}\noindent\hrulefill\par}

\newenvironment{block}[2][]
{\step[#1]  #2 \begin{\STEPS}}{\end{\STEPS}}

\newenvironment{case}[2][]
{\step[#1] (case \currentlabel) \emph{#2}\begin{\STEPS}}{\end{\STEPS}}

\newcommand{\setHeader}{\markboth
{\footnotesize CS 145 turn-in \turnIn, \today, by \author, \SID}
{\footnotesize CS 145 turn-in \turnIn, \today, by \author, \SID}}

\newcommand{\setTurnIn}[1]{\newcommand{\turnIn}{#1}}
\newcommand{\setAuthor}[1]{\renewcommand{\author}{#1}}
\newcommand{\setSID}[1]{\newcommand{\SID}{#1}}

\pagestyle{myheadings}
\addtolength{\headsep}{0.3in}
\addtolength{\topmargin}{-0.4in}
\addtolength{\textheight}{0.4in}

%%% Local Variables:
%%% mode: latex
%%% TeX-master: "assignment-template"
%%% End:


\setTurnIn{extra 1}
\setAuthor{Stanley Cohen (scohe001)}
\setSID{861114309}
\setHeader

\begin{document}

\begin{problems}

  %%%%%%%%%%%%%%%%%%%%%%%%%%%%%%%%%%%%%%%%%%%%%%%%%%%%%%%% 5

  \newpage

  \problem

  \begin{theorem}[from the proof-guide handout]
      25 boys and 25 girls sit around a table. It is always possible to find a person 
      that sits next to two girls (one on each side of the person).
  \end{theorem}

  \smallskip

  Note: I'm assuming here that the seats have been numbered from $1-50$ where 
  consecutive integers are seats that are next to each other and seat $50$ is next to seat $1$

  \begin{longFormProof}

    \begin{block}[A]
     {Assume for contradiction we have some ordering where there is no person who has a girl on both sides of them.}

      \step Call the odd numbered chairs $S_o$ and the evens $S_e$.
      
      \step In each of these sets there cannot be two consecutive girls.


    
        \smallskip 
        \hrulefill
        \smallskip 

        \begin{lemma}
          In both $S_o$ and $S_e$ there are not two consecutive boys.
        \end{lemma}

        \begin{longFormProof}[Proof of lemma.]
          \begin{block}[B]
          {Assume for contradiction there are two consecutive boys in one of the even or odd sets.}

            \step Since there are the same number of girls and boys, to fill all of the seats there must be two consecutive girls in one of the sets.

            \step Contradiction by 1.2.

          \end{block}

          \step By block~\ref{B} there are no two consecutive boys in the odd or even sets.
        \end{longFormProof}

        \vspace*{-1em}
        \hrulefill 
        \smallskip


    \step By Lemma 1 there are no two boys next to each other

    \step By 1.2 and 1.3, $S_o$ and $S_e$ must both be ordered in the form $\{..., boy, girl, boy, girl, ...\}$

    \step Each set has 25 people in it, so one set has 12 girls and 13 boys while the other has 13 girls and 12 boys.

    \begin{case}[case1]
    {Suppose $S_o$ has 13 girls.}

    \step Because of 1.4, the set must begin and end with a girl.

    \step This means that since the table is circular, someone is sitting next to two girls.

    \step Contradiction.

    \end{case}


    \begin{case}[case2]
    {Suppose $S_e$ has 13 girls.}

    \step Because of 1.4, the set must begin and end with a girl.

    \step This means that since the table is circular, someone is sitting next to two girls.

    \step Contradiction.

    \end{case}

    \step Case~\ref{case1} or~\ref{case2} must hold, therefore we have a contradiction.

    \end{block}

    \step By block~\ref{A}, every arrangement of boys and girls will result in at least one person sitting next to two girls.
  \end{longFormProof}


\end{problems}
\end{document}

%%% Local Variables:
%%% mode: latex
%%% TeX-master: t
%%% End:
