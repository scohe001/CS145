\documentclass[10pt]{article}

\input{assignment-macros-2}

% TODO: CHANGE NAME AND SID TO YOURS:

\setTurnIn{11}  

\setAuthor{YOUR NAME HERE}
\setSID{YOUR SID HERE}
\setHeader

\begin{document}

\begin{problems}

  %%%%%%%%%%%%%%%%%%%%%%%%%%%%%%%%%%%%%%%%%%%%%%%%%%%%%%%%%%%% 
  %%%%%%%%%%%%%%%%%%%%%%%%%%%%%%%%%%%%%%%%%%%%%%%%%%%%%%%%%%%% 

  \problem % problem 1

  \includegraphics{graph}

  {\vskip 0.1in \noindent\bf Maximum flows:}

    We can get a maximum flow of 9 by giving the edges in the graph the following flows:\\
    $f(s, v) = 5,\\
     f(s, u) = 4,\\
     f(v, w) = 2,\\
     f(v, z) = 2,\\
     f(v, u) = 1,\\
     f(u, w) = 1,\\
     f(u, z) = 4,\\
     f(z, t) = 4,\\
     f(z, w) = 2,\\
     f(w, t) = 5\\$

  {\vskip 0.1in \noindent\bf Minimum cuts:}

    There are a few minimum cuts for this problem. You could cut the set of edges $\{(s, v), (s, u)\}$ for a cost of 9, 
        or you could cut $\{(s, u), (v, w), (v, u), (v, z)\}$ for a cost of 9.

  {\vskip 0.1in \noindent\bf Critical edges:}

    Critical edges here are all in the set $\{(s, v), (s, u), (v, w), (v, u), (v, z)\}$
  
  %%%%%%%%%%%%%%%%%%%%%%%%%%%%%%%%%%%%%%%%%%%%%%%%%%%%%%%%%%%% 
  %%%%%%%%%%%%%%%%%%%%%%%%%%%%%%%%%%%%%%%%%%%%%%%%%%%%%%%%%%%% 

  \newpage 

  \problem % problem 2

  %%%%%%%%%%%%%%%%%%%%%%%%%%%%%%%%%%%%%%%%%%%%%%%%%%%%%%%%%%%% 
  %%%%%%%%%%%%%%%%%%%%%%%%%%%%%%%%%%%%%%%%%%%%%%%%%%%%%%%%%%%% 

  \setcounter{lemma}{1}
  \begin{lemma}
    An edge $e$ is critical if and only if $f(e) = \text{cap}(e)$ in every maximum flow $f$.
  \end{lemma}

  \begin{longFormProof}

    \begin{block}[B]
      {Consider any $s$-$t$ flow network $G$ and edge $e$.

      \lineacross }
      
      \step First we will prove that, if $f(e) \ne \text{cap}(e)$ in some maximum flow $f$, then $e$ is not critical.

      \smallskip 
      
      \begin{block}[B1]
        {Assume that there exists max-flow $f$ such that $f(e) \ne \text{cap}(e)$.}

        \step Let $\epsilon = \text{cap}(e) - f(e) > 0$.  (Using here that $f(e) \le \text{cap}(e)$ and $f(e)\ne\text{cap}(e)$.)

        \step Let $G'$ be the graph obtained by reducing the capacity of $e$ by $\epsilon$.

        \step $\text{cap}(e)$ in $G'$ is greater than or equal to $f(e)$ in $G$.

        \step Therefore, the max flow, $f$ in $G$ is also feasible in $G'$.

        \step So $e$ is not critical.
      \end{block}

      \step[B2] By block~\ref{B1}, if $f(e) \ne \text{cap}(e)$ in some maximum flow $f$,
      then $e$ is not critical.

      \smallskip 

      \lineacross 

      \step Next we will prove that, if $e$ is not critical, then $f(e) \ne \text{cap}(e)$ in some maximum flow.

      \smallskip 

      \begin{block}[B3]
        {Assume that $e$ is not critical in $G$.}

        \step[S1] Let $\epsilon>0$ be such that reducing the capacity of $e$ by $\epsilon$ does not reduce the maximum flow value.  (Such an $\epsilon$ exists by the definition of ``critical'').

        \step Let $G'$ be the graph obtained by reducing the capacity of $e$ by $\epsilon$.

        \step By the definition of a critical edge, $G'$ will have the same maximum flow value as $G$.

        \step That is, the maximum flow in $G'$ could be applied to $G$ and $f(e)$ would be $\text{cap}(e) - \epsilon$.

        \step There is a max flow $f$ in $G$ with $f(e) \ne \text{cap}(e)$.
      \end{block}
      
      \step[B4] By block~\ref{B3}, if $e$ is not critical, then there exists a maximum flow $f$ in $G$
      with $f(e) \ne \text{cap}(e)$.

      \smallskip

      \lineacross 
    
      \step By lines~\ref{B2} and~\ref{B4},
      $e$ is critical if and only if $f(e) = \text{cap}(e)$ in every maximum flow $f$ in $G$.
    \end{block}

    \step By block~\ref{B},
    for any edge $e$ in any flow network,
    $e$ is critical iff $f(e) = \text{cap}(e)$ in every max flow $f$ in $G$.
  \end{longFormProof}

\end{problems}

\end{document}

%%% Local Variables:
%%% mode: latex
%%% TeX-master: t
%%% End:
