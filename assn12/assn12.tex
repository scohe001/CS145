\documentclass[10pt]{article}

\input{assignment-macros-2}

% TODO: CHANGE NAME AND SID TO YOURS:

\setTurnIn{12}  

\setAuthor{YOUR NAME HERE}
\setSID{YOUR SID HERE}
\setHeader

\begin{document}

%%%%%%%%%%%%%%%%%%%%%%%%%%%%%%%%%%%%%%%%%%%%%%%%%%%%%%%%%%%% 
%%%%%%%%%%%%%%%%%%%%%%%%%%%%%%%%%%%%%%%%%%%%%%%%%%%%%%%%%%%% 

% LEMMA 1

% IF YOU ARE GOING TO PROVE LEMMA 1, USE THE PART BELOW
% OTHERWISE DELETE IT AND USE THE FOLLOWING PART

\begin{lemma}
  An edge $e$ is critical if and only if there exists a minimum-capacity cut containing $e$.
\end{lemma}

\begin{longFormProof}

  \begin{block}[A]
    {Consider any $s$-$t$ flow network $G$ and edge $e$.}

    \step Let $c_{\min}$ be the minimum capacity of any $s-t$ cut in $G$.

    \step By the max-flow/min-cut theorem, the maximum $s-t$ flow value in $G$ is $c_{\min}$. 

    \smallskip

    \lineacross
    
    \step First we will prove that, if no minimum cut contains $e$, then $e$ is not critical. 

    \begin{block}[A1]
      {Assume that no minimum-capacity cut contains $e$.}

      \step That is, the value of any cut that contains $e$ will be a loose upper bound for the max $s-t$ flow value in $G$.

      \step Therefore, there exists some $\epsilon > 0$ where $e$ can be reduced by $\epsilon$ 
              and all cuts with $e$ will still not be minimum capacity cuts. 

      \step That is, the upper bound on the maximum $s-t$ flow in $G$ will not change.

      \step So by definition $e$ is not critical.
    \end{block}
    
    \step[A2] By block~\ref{A1}, if no minimum-capacity cut contains $e$, then $e$ is not critical.

    \smallskip 
    
    \lineacross 

    \step Next we will prove that, if some minimum-capacity cut contains $e$, then $e$ is critical. 

    \begin{block}[A3]
      {Suppose that some minimum-capacity cut $C$ (of capacity $c_{\min}$) contains $e$.}

      \step We will show that $e$ is critical.  

      \smallskip
      
      \begin{block}[A31]
        {Consider any $\epsilon>0$.}

        \step Let $G'$ be the flow network obtained from $G$ by reducing the capacity of $e$ by $\epsilon$.

        \step Call the minimum cut $C$ in $G'$, $C'$ of value $c'_{\min}$.

        \step $c'_{\min} < c_{\min}$ since the value of $e$ has been reduced.

        \step By definition every cut is an upper bound on the max $s-t$ flow through some graph.

        \step Since we have a cut in $G'$ whose value is less than any minimum cut in $G$, the max $s-t$ flow of $G'$ is strictly less than that of $G$.

        \step That is, reducing $e$'s capacity by $\epsilon$ reduces the max flow.
      \end{block}

      \medskip 

      \step By block~\ref{A31}, for all $\epsilon>0$,
      reducing $e$'s capacity by $\epsilon$ reduces the max flow.
      That is, $e$ is critical.
    \end{block}
    
    \step[A4] By block~\ref{A3}, if some minimum-capacity cut contains $e$, then $e$ is critical.
    
    \smallskip 

    \lineacross 

    \step By lines~\ref{A2} and~\ref{A4}, $e$ is critical if and only if $e$ is on some minimum-capacity cut. 
  \end{block}

  \step By block~\ref{A}, for any edge $e$ in any flow network, $e$ is critical iff $e$ is on some minimum-capacity cut. 
\end{longFormProof}


\end{document}

%%% Local Variables:
%%% mode: latex
%%% TeX-master: t
%%% End:
