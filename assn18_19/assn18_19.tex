\documentclass[11pt]{article}

\usepackage{xifthen}% \isempty
\usepackage{stringstrings}% \substring

\usepackage{nameref}
\makeatletter
\newcommand*{\currentlabel}{\@currentlabel}
\makeatother

\usepackage{amsmath, amsthm, amssymb}
\usepackage{fullpage}
\usepackage[colorlinks=false,urlcolor=blue,pageanchor=true]{hyperref}
\usepackage{xspace}
\usepackage{graphicx}
\usepackage[normalem]{ulem}
\usepackage{enumitem}
\usepackage{adjustbox}

\newtheorem{lemma}{Lemma}
\newtheorem{definition}{Definition}
\newtheorem{theorem}{Theorem}
\newtheorem{corollary}{Corollary}
\newtheorem{claim}{Claim}

\newcommand{\E}{\operatorname{E}}
\newcommand{\giv}{\,|\,}
\newcommand{\Z}{\mathbb{Z}}
\newcommand{\Zp}{\mathbb{Z}_{\ge 0}}
\newcommand{\R}{\mathbb{R}}
\newcommand{\Rp}{\mathbb{R}_{\ge 0}}
\newcommand{\N}{\mathbb{N}}
\newcommand{\Np}{\mathbb{N}_{\ge 0}}

\newcommand{\OP}[1]{\text{\small\sc #1}\xspace}
\newcommand{\Insert}{\OP{Insert}}
\newcommand{\FindMin}{\OP{Find-Min}}
\newcommand{\DeleteMin}{\OP{Delete-Min}}
\newcommand{\DecreaseKey}{\OP{Decrease}}
\newcommand{\Merge}{\OP{Merge}}
\newcommand{\Cut}{\OP{Cut}}
\newcommand{\prob}[1]{\text{\textsc{#1}}}
\newcommand{\encode}[1]{\langle #1 \rangle}

\setlength{\parskip}{2pt}
% \setlength{\parindent}{0in}

\setlength\fboxsep{0pt}
\newlist{steps}{enumerate}{9}
\setlist[steps]{wide,label*=\arabic*.,topsep=1pt,parsep=1pt,partopsep=1pt,itemsep=0pt, labelindent=2pt}%
\setlist[steps,1]{ref=\arabic*}
\setlist[steps,2]{ref=\arabic{stepsi}.\arabic*}
\setlist[steps,3]{ref=\arabic{stepsi}.\arabic{stepsii}.\arabic*}
\setlist[steps,4]{ref=\arabic{stepsi}.\arabic{stepsii}.\arabic{stepsiii}.\arabic*}
\setlist[steps,5]{ref=\arabic{stepsi}.\arabic{stepsii}.\arabic{stepsiii}.\arabic{stepsiv}.\arabic*}
\setlist[steps,6]{ref=\arabic{stepsi}.\arabic{stepsii}.\arabic{stepsiii}.\arabic{stepsiv}.\arabic{\stepsv}.\arabic*}
\setlist[steps,7]{ref=\arabic{stepsi}.\arabic{stepsii}.\arabic{stepsiii}.\arabic{stepsiv}. \arabic{\stepsv}.\arabic{\stepsvi}.\arabic*}
\setlist[steps,8]{ref=\arabic{stepsi}.\arabic{stepsii}.\arabic{stepsiii}.\arabic{stepsiv}. \arabic{\stepsv}.\arabic{\stepsvi}.\arabic{\stepsvii}.\arabic*}
\setlist[steps,9]{ref=\arabic{stepsi}.\arabic{stepsii}.\arabic{stepsiii}.\arabic{stepsiv}. \arabic{\stepsv}.\arabic{\stepsvi}. \arabic{\stepsvii}.\arabic{\stepsviii}.\arabic*}

\newlist{asteps}{enumerate}{6}
\setlist[asteps]{wide,topsep=1pt,parsep=1pt,partopsep=1pt,itemsep=0pt, labelindent=1em}%
\setlist[asteps,1]{
  label=\alph*.,
  ref=\alph*
}
\setlist[asteps,2]{
  label=\alph{astepsi}.\arabic*.,
  ref=\alph{astepsi}.\arabic*
}
\setlist[asteps,3]{
  label=\alph{astepsi}.\arabic{astepsii}.\arabic*.,ref=\alph{astepsi}.\arabic{astepsii}.\arabic*
}
\setlist[asteps,4]{
  label=\alph{astepsi}.\arabic{astepsii}.\arabic{astepsiii}.\arabic*.,
  ref=\alph{astepsi}.\arabic{astepsii}.\arabic{astepsiii}.\arabic*
}
\setlist[asteps,5]{
  label=\alph{astepsi}.\arabic{astepsii}.\arabic{astepsiii}.\arabic{astepsiv}.\arabic*.,
  ref=\alph{astepsi}.\arabic{astepsii}.\arabic{astepsiii}.\arabic{astepsiv}.\arabic*
}
\setlist[asteps,6]{
  label=\alph{astepsi}.\arabic{astepsii}.\arabic{astepsiii}.\arabic{astepsiv}.\arabic{\astepsv}.\arabic*.,
  ref=\alph{astepsi}.\arabic{astepsii}.\arabic{astepsiii}.\arabic{astepsiv}.\arabic{\astepsv}.\arabic*
}

\newlist{problems}{enumerate}{9}
\setlist[problems]{wide, label=\textbf{Problem \arabic*},topsep=1pt,parsep=1pt,partopsep=1pt,itemsep=0pt, labelindent=0pt}%

\newcommand{\STEPS}{steps}
\newcommand{\SHIFT}{\renewcommand{\SHIFT}{\renewcommand{\STEPS}{asteps}}}

\newcommand{\maybeLabel}[1]{\ifthenelse{\isempty{#1}}{}{\label{#1}}}
\newcommand{\maybeItem}[1]{\ifthenelse{\isempty{#1}}{\item}{\item[#1]}}

\newcommand{\problem}[1][]{\maybeItem{#1}~\par}

\newenvironment{longFormProof}[1][Proof (long form).]
{\SHIFT\begin{proof}[#1]~\par\begin{\STEPS}}{\end{\STEPS}\vspace*{-1ex}\end{proof}}

\newenvironment{shortFormProof}[1][Proof (short form).]
{\SHIFT\begin{proof}[#1]}{\end{proof}}

% \makeatletter
% \newcommand{\caseitem}{\@sitem}
% \newcommand{\@sitem}{%
%   \refstepcounter{\@enumctr}%
%   \item[\textbullet~\csname label\@enumctr\endcsname]}
% \makeatother

\newcommand{\step}[1][]{\item\maybeLabel{#1}}

\newcommand{\comment}[1]{\hfill{\footnotesize\emph{#1}}}

\newcommand{\lineacross}{\par\vspace*{-0.7\baselineskip}\noindent\hrulefill\par}

\newenvironment{block}[2][]
{\step[#1]  #2 \begin{\STEPS}}{\end{\STEPS}}

\newenvironment{case}[2][]
{\step[#1] (case \currentlabel) \emph{#2}\begin{\STEPS}}{\end{\STEPS}}

\newcommand{\setHeader}{\markboth
{\footnotesize CS 145 turn-in \turnIn, \today, by \author, \SID}
{\footnotesize CS 145 turn-in \turnIn, \today, by \author, \SID}}

\newcommand{\setTurnIn}[1]{\newcommand{\turnIn}{#1}}
\newcommand{\setAuthor}[1]{\renewcommand{\author}{#1}}
\newcommand{\setSID}[1]{\newcommand{\SID}{#1}}

\pagestyle{myheadings}
\addtolength{\headsep}{0.3in}
\addtolength{\topmargin}{-0.4in}
\addtolength{\textheight}{0.4in}

%%% Local Variables:
%%% mode: latex
%%% TeX-master: "assignment-template"
%%% End:


% TODO: CHANGE NAME AND SID TO YOURS:

\setTurnIn{17/18}  
\setAuthor{Stanley Cohen (scohe001)}
\setSID{861114309}
\setHeader

\newcommand{\PROB}[2]{
  \begin{description}
  \item[]Definition of \prob{#1}:#2
  \end{description}
}

\begin{document}

\begin{problems}

  %%%%%%%%%%%%%%%%%%%%%%%%%%%%%%%%%%%%%%%%%%%%%%%%%%%%%%%%%%%% 
  %%%%%%%%%%%%%%%%%%%%%%%%%%%%%%%%%%%%%%%%%%%%%%%%%%%%%%%%%%%% 

  \problem % problem 1

    Describe a polynomial-time reduction from Vertex Cover to Set Cover.

  Given an instance $\encode{G=(V,E),k}$ of \prob{Vertex Cover},
  the reduction outputs the instance $\encode{U,C,k'}$ of \prob{Set Cover},
  where $U$, $C$, and $k'$ are defined as follows:

  \lineacross 

  Let $U = E$ be the set of edges in $G$;

  let $C_i$ be the edges touched by $v_i$, that is $C_i = \{e_{v_iv_j} | e_{v_iv_j} \in E\}$, unioned with $v_i$ itself;
  
  Let $k'$ be $k$ from the Vertex Cover problem.
  
  \lineacross 

  \medskip 
  Here is why the reduction can be computed in polynomial time:
  
  \lineacross 

  Constructing $C$ will require at worst a loop over the edge set for each vertex, resulting in $O(VE)$ time, which is polynomial.

  \lineacross 
  
  \medskip 
  Here is a proof that the reduction is correct.
  \begin{lemma}
    Given any instance $\encode{G,k}$ of \prob{Vertex Cover},
    let $\encode{U,C,k}$ be the instance of \prob{Set Cover} produced by the reduction.
    Then $G$ has a vertex cover of size $k$
    if and only if there is a set cover, $C'$ of size $k'$.
  \end{lemma}
  \begin{longFormProof}
    \step First we show the ``only if'' direction.
    \begin{block}[1oi]
      {Assume that $G$ has a vertex cover of size $k$.}
      \step Let $I$ be the vertex cover of size $k$ in $G$.
      \smallskip 

      \step Take $C'$ to be the set of sets in $C$ corresponding to the vertices in $I$.

      \step Then $C'$ has size $k'$.

      \step Since every edge, $v_iv_j$ in $E$, has a vertex in $I$ (that is, either $v_i$ or $v_j$ is in $I$ by definition of vertex cover), 
              all of the edges will be present in the union of the sets of $C'$.

      \step That is, the union of the sets of $C'$ will produce a set of all the edges in the graph, which is exactly $U$.

      \smallskip 
      \step So there is a subset of size $k$ of $C$ such that the union over them all will equal $U$.
    \end{block} 
    \step Next we show the ``if'' direction.
    \begin{block}[1i]
      {Assume that there are some $k'$ sets of $C$ such that the union over them all will equal $U$.}
      \step Let $C'$ be the set of $k'$ sets of $C$.
      \smallskip 

      \step Take $I$ to be the vertices corresponding to sets in $C'$.

      \step Then $I$ has size $k$.

      \step The union of the edges touched by the nodes in $I$ will be $U$, which is exactly $E$.

      \smallskip 
      \step There is a vertex cover of size $k=k'$ in $G$. 
    \end{block} 
    \step By blocks~\ref{1oi} and~\ref{1i},
    $G$ has a vertex cover of size $k$
    if and only if there is a set cover, $C'$ of size $k'$.
  \end{longFormProof}

  %%%%%%%%%%%%%%%%%%%%%%%%%%%%%%%%%%%%%%%%%%%%%%%%%%%%%%%%%%%% 
  %%%%%%%%%%%%%%%%%%%%%%%%%%%%%%%%%%%%%%%%%%%%%%%%%%%%%%%%%%%% 
  \newpage
  \problem % problem 2

  Describe a polynomial-time reduction from \prob{Subset Sum} to \prob{Zero Path}. 

  \PROB{Zero Path}{
    The input is $\langle G, s, t \rangle$,
    where $G=(V,E)$ is an acyclic directed graph with (possibly negative) integer edge weights,
    and $s$ and $t$ are vertices.
    The output is `yes' if there is a path $P$ from $s$ to $t$ in $G$ such that the total weight of the edges in $P$ is zero,
    and `no' otherwise.}

  \PROB{Subset Sum}{
    The input is $\langle x, T\rangle$,
    where $x=(x_1, x_2, \ldots, x_n)$ is a sequence of $n$ integers,  and $T$ is an integer.
    The output is `yes' if there is a subset $S\subseteq \{1,2,\ldots,n\}$
    with $\sum_{i\in S} x_i = T$, and `no' otherwise.}

  \lineacross 
  Create two vertices, $v_s$ and $v_{s'}$ with an edge, $v_sv_{s'}$ of weight, $-T$.

  Then create a widget for each $x_i$ in $x$ with two vertices, $v_i$ and $v_i'$ and two edges between them, $v_iv_i'$ of weights 0 and $x_i$.

  String together any two widgets, $i$ and $j$ by creating an edge $v_i'v_j$ of weight 0.

  Connect the first widget, $k$ with the start by creating the edge $v_{s'}v_k$ of weight 0.

  Connect the last widget, $l$ to the the end node, $v_t$ by adding the edge $v_l'v_t$ of weight 0.
  \lineacross 

  \medskip 
  Here is why the reduction can be computed in polynomial time:

  \lineacross 
  The reduction as described makes a single pass over the set $x$ to construct the graph, so it'll work in polynomial time.
  \lineacross 

  \medskip 
  Here is a proof that the reduction is correct.
  \begin{lemma}
    Given any instance $\encode{x, T}$ of \prob{Subset Sum},
    let $\encode{G=(V,E), s, t}$ be the instance of \prob{Zero Path} produced by the reduction.
    Then $x$ has a Subset Sum
    if and only if $G$ has a Zero Path.
  \end{lemma}
  \begin{longFormProof}
    \step First we show the ``only if'' direction.
    \begin{block}[2oi]
      {Assume that $x$ has a Subset Sum of $T$.}
      \smallskip 
      
      \step For each $x_i \in x$, if $x_i$ was used in the sum to get $T$, take the edge in the corresponding widget of weight $x_i$. 
              Else, take the edge of weight 0.

      \step Among the widgets traversed, the weight of the path will be $T$, but the first edge taken, 
              edge $v_sv_{s'}$ was of weight $-T$, so the path weight will be 0.

      \step So $G$ has a path from $v_s$ to $v_t$ of weight 0.

      \smallskip 
    \end{block} 
    \step Next we show the ``if'' direction.
    \begin{block}[2i]
      {Assume that $G$ has a Zero Path from $v_s$ to $v_t$.}
      \step Let $P$ be such a path.
      \smallskip 

      \step $P$ must traverse the edge $v_sv_{s'}$ as it is the only edge out of $v_s$.

      \step $P$ must traverse every widget since there is no other way to reach $v_t$ from $v_s$.

      \step At every widget, the weight of the path either increased by some $x_i$, or remained unchanged.

      \step For the weight of $P$ to be Zero, some set $x'$ of non-zero widget edges must've been traversed that sum to the weight of $v_sv_{s'}$.

      \step Then $x'$ is the set of widget weight that sum to $v_sv_{s'}$, which is exactly $T$.

      \step Further, the non-zero widget weights, $x' \subseteq x$.

      \step There is a subset, $x' \subseteq x$ whose values sum to $T$.

      \smallskip 
    \end{block} 
    \step By blocks~\ref{2oi} and~\ref{2i},
    $x$ has a Subset Sum
    if and only if $G$ has a Zero Path.
  \end{longFormProof}

  % %%%%%%%%%%%%%%%%%%%%%%%%%%%%%%%%%%%%%%%%%%%%%%%%%%%%%%%%%%%% 
  % %%%%%%%%%%%%%%%%%%%%%%%%%%%%%%%%%%%%%%%%%%%%%%%%%%%%%%%%%%%% 

  % \newpage
  % \problem % problem 3

  % Given an instance $\encode{G=(V,E)}$ of \prob{Directed Ham Cycle},
  % the reduction outputs the instance $\encode{\Phi}$ of \prob{SAT}, defined as follows:
  
  % \lineacross 

  % Let $n=|V|$.
  % The Boolean variables for the formula $\Phi$ are $X=\{x_{ij} : i\in V, j\in\{1,2,\ldots,n\}\}$,
  % with the interpretation that setting $x_{ij}$ to be true should correspond to having
  % a Hamiltonian cycle in $G$ with vertex $i$ at position $j$.

  % \newcommand{\one}{\mbox{\sf one}}
  % For any set of variables $S\subseteq X$, define $\one(S)$
  % to be the formula $(\bigvee_{x \in S} x) \wedge \neg (\bigvee_{x,x'\in S: x\ne x'} x \wedge \ x')$.
  % So $\one(S)$ is satisfied
  % if and only if exactly one of the variables in $S$ is assigned true.

  % Take $\Phi$ to be logical and of the following expressions:
  % \begin{enumerate}
  % \item For each vertex $i$, $\one(\{x_{i1}, x_{i2}, \ldots, x_{in}\})$.
  %   (Each vertex should have one position.)

  % \item For each position $j$, $\one(\{x_{1j}, x_{2j}, \ldots, x_{nj}\})$.
  %   (Each position should have one vertex.)

  % \item For each position $j$, and pair of vertices $i, i'$ such that $(i,i')\not\in E$,
  %   add clause $\neg (x_{ij} \wedge x_{i',j+1})$
  %   or, if $j=n$, then add clause  $\neg (x_{ij} \wedge x_{i',1})$.
  %   (In other words, vertex $i'$ cannot follow vertex $i$ on the cycle
  %   if there is no edge $(i,i')$).
  % \end{enumerate}
  
  % \lineacross 

  % \medskip 
  % Here is why the reduction can be computed in polynomial time:
  
  % \lineacross 
  
  % $\Phi$ can be built as described above in time $O(|V|^2 + |E|\,|V|)$
  % by appropriately enumerating the vertices $i\in V$ and positions $j\in\{1,\ldots,n\}$.

  % \lineacross 

  % \medskip 
  % Here is a proof that the reduction is correct.
  % \begin{lemma}
  %   Given any instance $\encode{G=(V,E)}$ of \prob{Directed Ham Cycle},
  %   let $\Phi$ be the instance of \prob{SAT} produced by the reduction.
  %   Then $G$ has a directed Hamiltonian cycle 
  %   if and only if $\Phi$ is satisfiable.
  % \end{lemma}
  % \begin{longFormProof}
  %   \step First we show the ``only if'' direction.
  %   \begin{block}[3oi]
  %     {Assume that $G$ has a directed Hamiltonian cycle.}
  %     \step Let $C$ be such a cycle.
  %     \smallskip 

  %     \emph{\ldots Somehow show that, since $C$ exists in $G$,
  %       there must be an assignment $A$ to the variables of $\Phi$
  %       that makes $\Phi$ true\ldots}

  %     \smallskip 
  %     \step $\Phi$ is satisfiable.
  %   \end{block} 
  %   \step Next we show the ``if'' direction.
  %   \begin{block}[3i]
  %     {Assume that $\Phi$ is satisfiable.}
  %     \step Let $A$ be an assignment to the variables of $\Phi$ that makes $\Phi$ true.
  %     \smallskip 
      
  %     \emph{\ldots Somehow show that, since $A$ exists,
  %       there must be a directed Hamiltonian cycle $C$ in $G$\ldots}

  %     \smallskip 
  %     \step There is an directed Hamiltonian cycle in $G$.
  %   \end{block} 
  %   \step By blocks~\ref{3oi} and~\ref{3i},
  %   $G$ has a directed Hamiltonian cycle
  %   if and only if $\Phi$ is satisfiable.
  % \end{longFormProof}

  % %%%%%%%%%%%%%%%%%%%%%%%%%%%%%%%%%%%%%%%%%%%%%%%%%%%%%%%%%%%% 
  % %%%%%%%%%%%%%%%%%%%%%%%%%%%%%%%%%%%%%%%%%%%%%%%%%%%%%%%%%%%% 

\end{problems}

\end{document}

%%% Local Variables:
%%% mode: latex
%%% TeX-master: t
%%% End:
