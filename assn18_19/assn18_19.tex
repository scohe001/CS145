\documentclass[11pt]{article}

\usepackage{xifthen}% \isempty
\usepackage{stringstrings}% \substring

\usepackage{nameref}
\makeatletter
\newcommand*{\currentlabel}{\@currentlabel}
\makeatother

\usepackage{amsmath, amsthm, amssymb}
\usepackage{fullpage}
\usepackage[colorlinks=false,urlcolor=blue,pageanchor=true]{hyperref}
\usepackage{xspace}
\usepackage{graphicx}
\usepackage[normalem]{ulem}
\usepackage{enumitem}
\usepackage{adjustbox}

\newtheorem{lemma}{Lemma}
\newtheorem{definition}{Definition}
\newtheorem{theorem}{Theorem}
\newtheorem{corollary}{Corollary}
\newtheorem{claim}{Claim}

\newcommand{\E}{\operatorname{E}}
\newcommand{\giv}{\,|\,}
\newcommand{\Z}{\mathbb{Z}}
\newcommand{\Zp}{\mathbb{Z}_{\ge 0}}
\newcommand{\R}{\mathbb{R}}
\newcommand{\Rp}{\mathbb{R}_{\ge 0}}
\newcommand{\N}{\mathbb{N}}
\newcommand{\Np}{\mathbb{N}_{\ge 0}}

\newcommand{\OP}[1]{\text{\small\sc #1}\xspace}
\newcommand{\Insert}{\OP{Insert}}
\newcommand{\FindMin}{\OP{Find-Min}}
\newcommand{\DeleteMin}{\OP{Delete-Min}}
\newcommand{\DecreaseKey}{\OP{Decrease}}
\newcommand{\Merge}{\OP{Merge}}
\newcommand{\Cut}{\OP{Cut}}
\newcommand{\prob}[1]{\text{\textsc{#1}}}
\newcommand{\encode}[1]{\langle #1 \rangle}

\setlength{\parskip}{2pt}
% \setlength{\parindent}{0in}

\setlength\fboxsep{0pt}
\newlist{steps}{enumerate}{9}
\setlist[steps]{wide,label*=\arabic*.,topsep=1pt,parsep=1pt,partopsep=1pt,itemsep=0pt, labelindent=2pt}%
\setlist[steps,1]{ref=\arabic*}
\setlist[steps,2]{ref=\arabic{stepsi}.\arabic*}
\setlist[steps,3]{ref=\arabic{stepsi}.\arabic{stepsii}.\arabic*}
\setlist[steps,4]{ref=\arabic{stepsi}.\arabic{stepsii}.\arabic{stepsiii}.\arabic*}
\setlist[steps,5]{ref=\arabic{stepsi}.\arabic{stepsii}.\arabic{stepsiii}.\arabic{stepsiv}.\arabic*}
\setlist[steps,6]{ref=\arabic{stepsi}.\arabic{stepsii}.\arabic{stepsiii}.\arabic{stepsiv}.\arabic{\stepsv}.\arabic*}
\setlist[steps,7]{ref=\arabic{stepsi}.\arabic{stepsii}.\arabic{stepsiii}.\arabic{stepsiv}. \arabic{\stepsv}.\arabic{\stepsvi}.\arabic*}
\setlist[steps,8]{ref=\arabic{stepsi}.\arabic{stepsii}.\arabic{stepsiii}.\arabic{stepsiv}. \arabic{\stepsv}.\arabic{\stepsvi}.\arabic{\stepsvii}.\arabic*}
\setlist[steps,9]{ref=\arabic{stepsi}.\arabic{stepsii}.\arabic{stepsiii}.\arabic{stepsiv}. \arabic{\stepsv}.\arabic{\stepsvi}. \arabic{\stepsvii}.\arabic{\stepsviii}.\arabic*}

\newlist{asteps}{enumerate}{6}
\setlist[asteps]{wide,topsep=1pt,parsep=1pt,partopsep=1pt,itemsep=0pt, labelindent=1em}%
\setlist[asteps,1]{
  label=\alph*.,
  ref=\alph*
}
\setlist[asteps,2]{
  label=\alph{astepsi}.\arabic*.,
  ref=\alph{astepsi}.\arabic*
}
\setlist[asteps,3]{
  label=\alph{astepsi}.\arabic{astepsii}.\arabic*.,ref=\alph{astepsi}.\arabic{astepsii}.\arabic*
}
\setlist[asteps,4]{
  label=\alph{astepsi}.\arabic{astepsii}.\arabic{astepsiii}.\arabic*.,
  ref=\alph{astepsi}.\arabic{astepsii}.\arabic{astepsiii}.\arabic*
}
\setlist[asteps,5]{
  label=\alph{astepsi}.\arabic{astepsii}.\arabic{astepsiii}.\arabic{astepsiv}.\arabic*.,
  ref=\alph{astepsi}.\arabic{astepsii}.\arabic{astepsiii}.\arabic{astepsiv}.\arabic*
}
\setlist[asteps,6]{
  label=\alph{astepsi}.\arabic{astepsii}.\arabic{astepsiii}.\arabic{astepsiv}.\arabic{\astepsv}.\arabic*.,
  ref=\alph{astepsi}.\arabic{astepsii}.\arabic{astepsiii}.\arabic{astepsiv}.\arabic{\astepsv}.\arabic*
}

\newlist{problems}{enumerate}{9}
\setlist[problems]{wide, label=\textbf{Problem \arabic*},topsep=1pt,parsep=1pt,partopsep=1pt,itemsep=0pt, labelindent=0pt}%

\newcommand{\STEPS}{steps}
\newcommand{\SHIFT}{\renewcommand{\SHIFT}{\renewcommand{\STEPS}{asteps}}}

\newcommand{\maybeLabel}[1]{\ifthenelse{\isempty{#1}}{}{\label{#1}}}
\newcommand{\maybeItem}[1]{\ifthenelse{\isempty{#1}}{\item}{\item[#1]}}

\newcommand{\problem}[1][]{\maybeItem{#1}~\par}

\newenvironment{longFormProof}[1][Proof (long form).]
{\SHIFT\begin{proof}[#1]~\par\begin{\STEPS}}{\end{\STEPS}\vspace*{-1ex}\end{proof}}

\newenvironment{shortFormProof}[1][Proof (short form).]
{\SHIFT\begin{proof}[#1]}{\end{proof}}

% \makeatletter
% \newcommand{\caseitem}{\@sitem}
% \newcommand{\@sitem}{%
%   \refstepcounter{\@enumctr}%
%   \item[\textbullet~\csname label\@enumctr\endcsname]}
% \makeatother

\newcommand{\step}[1][]{\item\maybeLabel{#1}}

\newcommand{\comment}[1]{\hfill{\footnotesize\emph{#1}}}

\newcommand{\lineacross}{\par\vspace*{-0.7\baselineskip}\noindent\hrulefill\par}

\newenvironment{block}[2][]
{\step[#1]  #2 \begin{\STEPS}}{\end{\STEPS}}

\newenvironment{case}[2][]
{\step[#1] (case \currentlabel) \emph{#2}\begin{\STEPS}}{\end{\STEPS}}

\newcommand{\setHeader}{\markboth
{\footnotesize CS 145 turn-in \turnIn, \today, by \author, \SID}
{\footnotesize CS 145 turn-in \turnIn, \today, by \author, \SID}}

\newcommand{\setTurnIn}[1]{\newcommand{\turnIn}{#1}}
\newcommand{\setAuthor}[1]{\renewcommand{\author}{#1}}
\newcommand{\setSID}[1]{\newcommand{\SID}{#1}}

\pagestyle{myheadings}
\addtolength{\headsep}{0.3in}
\addtolength{\topmargin}{-0.4in}
\addtolength{\textheight}{0.4in}

%%% Local Variables:
%%% mode: latex
%%% TeX-master: "assignment-template"
%%% End:


% TODO: CHANGE NAME AND SID TO YOURS:

\setTurnIn{17 solutions}  
\setAuthor{instructor}
\setSID{INCOMPLETE -- TO BE COMPLETED LATER THIS EVENING}
\setHeader

\begin{document}

\begin{problems}

  %%%%%%%%%%%%%%%%%%%%%%%%%%%%%%%%%%%%%%%%%%%%%%%%%%%%%%%%%%%% 
  %%%%%%%%%%%%%%%%%%%%%%%%%%%%%%%%%%%%%%%%%%%%%%%%%%%%%%%%%%%% 

  \problem % problem 1

  Given an instance $\encode{G=(V,E),k}$ of \prob{Independent Set},
  the reduction outputs the instance $\encode{n, D, p, i}$ of \prob{ECD},
  where $n$, $D$, $p$, and $i$ are defined as follows:

  \lineacross 

  Let $n= |V|$ be the number of vertices in $G$;
  let $D_{ij}$ be one if there is an edge $(i,j)\in E$ and zero otherwise (for $i,j\in\{1,2,\ldots,n\}$);
  Let $p=0$ and $i=k$.
  
  \lineacross 

  \medskip 
  Here is why the reduction can be computed in polynomial time:
  
  \lineacross 

  Given $G$ and $k$, we can compute $n=|V|$ in linear time by counting the vertices.
  We can compute the matrix $D$ in $O(n^2)$ time in a single pass over the edges.
  We can compute $p$ and $k$ in constant time.

  \lineacross 
  
  \medskip 
  Here is a proof that the reduction is correct.
  \begin{lemma}
    Given any instance $\encode{G,k}$ of \prob{Independent Set},
    let $\encode{n, D, p, i}$ be the instance of \prob{ECD} produced by the reduction.
    Then $G$ has an independent set of size $k$
    if and only if there is a set of $i$ ingredients that have total discord at most $p$.
  \end{lemma}
  \begin{longFormProof}
    \step First we show the ``only if'' direction.
    \begin{block}[1oi]
      {Assume that $G$ has an independent set of size $k$.}
      \step Let $I$ be an independent set of size $k$ in $G$.
      \smallskip 

      \step Take $S$ to be the set of ingredients corresponding to vertices in $I$.

      \step Then $S$ has size $k$, and has total discord zero
      because each pair $(i,j)$ of ingredients in $S$ corresponds to a pair of vertices
      with no edge ($D_{ij}=0$).

      \smallskip 
      \step So there is a set of $i=k$ ingredients that have total discord at most $p=0$.
    \end{block} 
    \step Next we show the ``if'' direction.
    \begin{block}[1i]
      {Assume there is a set of at least $i$ ingredients that have total discord at most $p$.}
      \step Let $S$ be a set of at $i$ ingredients that have total discord at most $p$.  Note $i=k$ and $p=0$.
      \smallskip 

      \step Take $I$ to be the vertices corresponding to ingredients in $S$.

      \step Then $I$ has size $i$, and is an independent set because
      each pair $(i,j)$ of vertices in $I$ corresponds to a pair of ingredients
      with zero discord ($D_{ij}\le p=0$), so there is no edge $(i,j)$.

      \smallskip 
      \step There is an independent set of size $k=i$ in $G$. 
    \end{block} 
    \step By blocks~\ref{1oi} and~\ref{1i},
    $G$ has an independent set of size $k$
    if and only if there is a set of $i$ ingredients that have total discord at most $p$.
  \end{longFormProof}

  %%%%%%%%%%%%%%%%%%%%%%%%%%%%%%%%%%%%%%%%%%%%%%%%%%%%%%%%%%%% 
  %%%%%%%%%%%%%%%%%%%%%%%%%%%%%%%%%%%%%%%%%%%%%%%%%%%%%%%%%%%% 
  \newpage
  \problem % problem 2

  Given an instance $\encode{G=(V,E)}$ of \prob{Directed Ham Cycle},
  the reduction outputs the instance $\encode{G'=(V',E')}$ of \prob{Undirected Ham Cycle},
  where $V'$ and $E'$ are defined as follows:

  \lineacross 
  For each vertex $v\in V$, add three \emph{clone} vertices $v^1, v^2, v^3$ to $V'$;
  add edges $(v^1,v^2)$ and $(v^2, v^3)$ to $E'$.
  For each edge $(u,w) \in E$, add edge $(w^3, v^1)$ to $E'$.
  \lineacross 

  \medskip 
  Here is why the reduction can be computed in polynomial time:

  \lineacross 
  The reduction as described makes a single pass over the vertices and edges
  of $G$ to build $G'$.  So it takes linear time.
  \lineacross 

  \medskip 
  Here is a proof that the reduction is correct.
  \begin{lemma}
    Given any instance $\encode{G=(V,E)}$ of \prob{Directed Ham Cycle},
    let $\encode{G'=(V',E')}$ be the instance of \prob{Undirected Ham Cycle} produced by the reduction.
    Then $G$ has a directed Hamiltonian cycle 
    if and only if $G'$ has an undirected Hamiltonian cycle.
  \end{lemma}
  \begin{longFormProof}
    \step First we show the ``only if'' direction.
    \begin{block}[2oi]
      {Assume that $G$ has a directed Hamiltonian cycle.}
      \step Let $C=(v_1, v_2, \ldots, v_n, v_1)$ be such a cycle.
      \smallskip 
      
      \step Obtain $C'$ from $C$ by replacing each vertex $v_i$ by its three clones $v^1, v^2, v^3$.

      \step So $C'=(v^1_1, v^2_1 v^3_1, v^1_2, v^2_2, v^3_2, \ldots, v^1_n,  v^2_n,  v^3_n,  v^1_1,  v^2_1,  v^3_1)$.

      \step Since $C$ is a Hamiltonian cycle, each edge $(v_i, v_{i+1})$ is present in $G$,
      and each vertex $v$ in $G$ is visited exactly once by $C$.

      \step So each edge $(v^3_i, v^1_{i+1})$ is present in $G'$ (by the reduction),
      and each vertex $v_i^\ell$ is visited exactly once by $C'$.

      \step Also, for each $i$, each edge $(v^1_i, v^2_i)$ and $(v^2_i, v^3_i)$ is present in $G'$ (by the reduction). 

      \step So $C'$ is a Hamiltonian cycle in $G'$.

      \smallskip 
    \end{block} 
    \step Next we show the ``if'' direction.
    \begin{block}[2i]
      {Assume that $G'$ has a undirected Hamiltonian cycle.}
      \step Let $C'$ be such a cycle.
      \smallskip 

      \step For each $v\in V$, the vertex $v^2$ has edges only to $v^1$ and $v^3$,
      so edges $(v^1, v^2)$ and $(v^2, v^3)$ are present in $C'$.
      Further, each $v^3$ has edges only to vertices $w^1$ such that $(v,w)\in E$.

      \step So (starting the cycle $C'$ at an arbitrary vertex $v^1$, and orienting it so that
      $v^2$ is the second vertex on the cycle), by a simple induction, the $3n$ vertices in $G'$ can be ordered so that

      \[C'=(v^1_1, v^2_1 v^3_1, v^1_2, v^2_2, v^3_2, \ldots, v^1_n,  v^2_n,  v^3_n,  v^1_1,  v^2_1,  v^3_1).\]

      \step Let $C=(v_1, v_2, \ldots, v_n, v_1)$ be obtained by replacing each clone triple $(v_i^1, v_i^2, v_i^3)$ in $V'$
      by its original vertex $v_i$ in $G$.
      So each vertex $v$ in $G$ is visited exactly once by $C$.

      \step Each edge $(v^3_i, v_{i+1}^1)$ is present in $G'$, so, by the reduction,
      each edge $(v_i, v_{i+1}$ is present in $G$.

      \step So $C$ is a Hamiltonian cycle in $G$.

      \smallskip 
    \end{block} 
    \step By blocks~\ref{2oi} and~\ref{2i},
    $G$ has a directed Hamiltonian cycle
    if and only if $G'$ has an undirected Hamiltonian cycle.
  \end{longFormProof}

  %%%%%%%%%%%%%%%%%%%%%%%%%%%%%%%%%%%%%%%%%%%%%%%%%%%%%%%%%%%% 
  %%%%%%%%%%%%%%%%%%%%%%%%%%%%%%%%%%%%%%%%%%%%%%%%%%%%%%%%%%%% 

  \newpage
  \problem % problem 3

  Given an instance $\encode{G=(V,E)}$ of \prob{Directed Ham Cycle},
  the reduction outputs the instance $\encode{\Phi}$ of \prob{SAT}, defined as follows:
  
  \lineacross 

  Let $n=|V|$.
  The Boolean variables for the formula $\Phi$ are $X=\{x_{ij} : i\in V, j\in\{1,2,\ldots,n\}\}$,
  with the interpretation that setting $x_{ij}$ to be true should correspond to having
  a Hamiltonian cycle in $G$ with vertex $i$ at position $j$.

  \newcommand{\one}{\mbox{\sf one}}
  For any set of variables $S\subseteq X$, define $\one(S)$
  to be the formula $(\bigvee_{x \in S} x) \wedge \neg (\bigvee_{x,x'\in S: x\ne x'} x \wedge \ x')$.
  So $\one(S)$ is satisfied
  if and only if exactly one of the variables in $S$ is assigned true.

  Take $\Phi$ to be logical and of the following expressions:
  \begin{enumerate}
  \item For each vertex $i$, $\one(\{x_{i1}, x_{i2}, \ldots, x_{in}\})$.
    (Each vertex should have one position.)

  \item For each position $j$, $\one(\{x_{1j}, x_{2j}, \ldots, x_{nj}\})$.
    (Each position should have one vertex.)

  \item For each position $j$, and pair of vertices $i, i'$ such that $(i,i')\not\in E$,
    add clause $\neg (x_{ij} \wedge x_{i',j+1})$
    or, if $j=n$, then add clause  $\neg (x_{ij} \wedge x_{i',1})$.
    (In other words, vertex $i'$ cannot follow vertex $i$ on the cycle
    if there is no edge $(i,i')$).
  \end{enumerate}
  
  \lineacross 

  \medskip 
  Here is why the reduction can be computed in polynomial time:
  
  \lineacross 
  
  $\Phi$ can be built as described above in time $O(|V|^2 + |E|\,|V|)$
  by appropriately enumerating the vertices $i\in V$ and positions $j\in\{1,\ldots,n\}$.

  \lineacross 

  \medskip 
  Here is a proof that the reduction is correct.
  \begin{lemma}
    Given any instance $\encode{G=(V,E)}$ of \prob{Directed Ham Cycle},
    let $\Phi$ be the instance of \prob{SAT} produced by the reduction.
    Then $G$ has a directed Hamiltonian cycle 
    if and only if $\Phi$ is satisfiable.
  \end{lemma}
  \begin{longFormProof}
    \step First we show the ``only if'' direction.
    \begin{block}[3oi]
      {Assume that $G$ has a directed Hamiltonian cycle.}
      \step Let $C$ be such a cycle.
      \smallskip 

      \emph{\ldots Somehow show that, since $C$ exists in $G$,
        there must be an assignment $A$ to the variables of $\Phi$
        that makes $\Phi$ true\ldots}

      \smallskip 
      \step $\Phi$ is satisfiable.
    \end{block} 
    \step Next we show the ``if'' direction.
    \begin{block}[3i]
      {Assume that $\Phi$ is satisfiable.}
      \step Let $A$ be an assignment to the variables of $\Phi$ that makes $\Phi$ true.
      \smallskip 
      
      \emph{\ldots Somehow show that, since $A$ exists,
        there must be a directed Hamiltonian cycle $C$ in $G$\ldots}

      \smallskip 
      \step There is an directed Hamiltonian cycle in $G$.
    \end{block} 
    \step By blocks~\ref{3oi} and~\ref{3i},
    $G$ has a directed Hamiltonian cycle
    if and only if $\Phi$ is satisfiable.
  \end{longFormProof}

  %%%%%%%%%%%%%%%%%%%%%%%%%%%%%%%%%%%%%%%%%%%%%%%%%%%%%%%%%%%% 
  %%%%%%%%%%%%%%%%%%%%%%%%%%%%%%%%%%%%%%%%%%%%%%%%%%%%%%%%%%%% 

\end{problems}

\end{document}

%%% Local Variables:
%%% mode: latex
%%% TeX-master: t
%%% End:
