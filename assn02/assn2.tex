\documentclass[11pt]{article}

\usepackage{xifthen}% \isempty
\usepackage{stringstrings}% \substring

\usepackage{nameref}
\makeatletter
\newcommand*{\currentlabel}{\@currentlabel}
\makeatother

\usepackage{amsmath, amsthm, amssymb}
\usepackage{fullpage}
\usepackage[colorlinks=false,urlcolor=blue,pageanchor=true]{hyperref}
\usepackage{xspace}
\usepackage{graphicx}
\usepackage[normalem]{ulem}
\usepackage[shortlabels]{enumitem}

\newtheorem{lemma}{Lemma}
\newtheorem{definition}{Definition}
\newtheorem{theorem}{Theorem}
\newtheorem{corollary}{Corollary}
\newtheorem{claim}{Claim}

\newcommand{\E}{\operatorname{E}}
\newcommand{\giv}{\,|\,}
\newcommand{\Z}{\mathbb{Z}}
\newcommand{\Zp}{\mathbb{Z}_{\ge 0}}
\newcommand{\R}{\mathbb{R}}
\newcommand{\Rp}{\mathbb{R}_{\ge 0}}
\newcommand{\N}{\mathbb{N}}
\newcommand{\Np}{\mathbb{N}_{\ge 0}}

\newcommand{\OP}[1]{{\small\sc #1}\xspace}
\newcommand{\Insert}{\OP{Insert}}
\newcommand{\FindMin}{\OP{Find-Min}}
\newcommand{\DeleteMin}{\OP{Delete-Min}}
\newcommand{\DecreaseKey}{\OP{Decrease}}
\newcommand{\Merge}{\OP{Merge}}
\newcommand{\Cut}{\OP{Cut}}

\setlength{\parskip}{2pt}
\setlength{\parindent}{0in}

\setlength\fboxsep{0pt}
\newlist{steps}{enumerate}{9}
\setlist[steps]{wide, label*=\arabic*.,topsep=1pt,parsep=1pt,partopsep=1pt,itemsep=0pt}%
\setlist[steps,1]{ref=\arabic*}
\setlist[steps,2]{ref=\arabic{stepsi}.\arabic*}
\setlist[steps,3]{ref=\arabic{stepsi}.\arabic{stepsii}.\arabic*}
\setlist[steps,4]{ref=\arabic{stepsi}.\arabic{stepsii}.\arabic{stepsiii}.\arabic*}
\setlist[steps,5]{ref=\arabic{stepsi}.\arabic{stepsii}.\arabic{stepsiii}.\arabic{stepsiv}.\arabic*}
\setlist[steps,6]{ref=\arabic{stepsi}.\arabic{stepsii}.\arabic{stepsiii}.\arabic{stepsiv}.\arabic{\stepsv}.\arabic*}
\setlist[steps,7]{ref=\arabic{stepsi}.\arabic{stepsii}.\arabic{stepsiii}.\arabic{stepsiv}. \arabic{\stepsv}.\arabic{\stepsvi}.\arabic*}
\setlist[steps,8]{ref=\arabic{stepsi}.\arabic{stepsii}.\arabic{stepsiii}.\arabic{stepsiv}. \arabic{\stepsv}.\arabic{\stepsvi}.\arabic{\stepsvii}.\arabic*}
\setlist[steps,9]{ref=\arabic{stepsi}.\arabic{stepsii}.\arabic{stepsiii}.\arabic{stepsiv}. \arabic{\stepsv}.\arabic{\stepsvi}. \arabic{\stepsvii}.\arabic{\stepsviii}.\arabic*}

\newlist{asteps}{enumerate}{9}
\setlist[asteps]{wide, label*=\alph*.,topsep=1pt,parsep=1pt,partopsep=1pt,itemsep=0pt}%
\setlist[asteps,1]{ref=\alph*}
\setlist[asteps,2]{ref=\alph{astepsi}.\alph*}
\setlist[asteps,3]{ref=\alph{astepsi}.\alph{astepsii}.\alph*}
\setlist[asteps,4]{ref=\alph{astepsi}.\alph{astepsii}.\alph{astepsiii}.\alph*}
\setlist[asteps,5]{ref=\alph{astepsi}.\alph{astepsii}.\alph{astepsiii}.\alph{astepsiv}.\alph*}
\setlist[asteps,6]{ref=\alph{astepsi}.\alph{astepsii}.\alph{astepsiii}.\alph{astepsiv}.\alph{\astepsv}.\alph*}
\setlist[asteps,7]{ref=\alph{astepsi}.\alph{astepsii}.\alph{astepsiii}.\alph{astepsiv}. \alph{\astepsv}.\alph{\astepsvi}.\alph*}
\setlist[asteps,8]{ref=\alph{astepsi}.\alph{astepsii}.\alph{astepsiii}.\alph{astepsiv}. \alph{\astepsv}.\alph{\astepsvi}.\alph{\astepsvii}.\alph*}
\setlist[asteps,9]{ref=\alph{astepsi}.\alph{astepsii}.\alph{astepsiii}.\alph{astepsiv}. \alph{\astepsv}.\alph{\astepsvi}. \alph{\astepsvii}.\alph{\astepsviii}.\alph*}

\newlist{problems}{enumerate}{9}
\setlist[problems]{wide, label=\textbf{Problem \arabic*},topsep=1pt,parsep=1pt,partopsep=1pt,itemsep=0pt}%

\newcommand{\STEPS}{steps}
\newcommand{\SHIFT}{\renewcommand{\SHIFT}{\renewcommand{\STEPS}{asteps}}}

\newcommand{\maybeLabel}[1]{\ifthenelse{\isempty{#1}}{}{\label{#1}}}
\newcommand{\maybeItem}[1]{\ifthenelse{\isempty{#1}}{\item}{\item[#1]}}

\newcommand{\problem}[1][]{\maybeItem{#1}~\par}

\newenvironment{longFormProof}[1][Proof (long form).]
{\SHIFT\begin{proof}[#1]~\par\begin{\STEPS}}{\end{\STEPS}\vspace*{-1ex}\end{proof}}

\newenvironment{shortFormProof}[1][Proof (short form).]
{\SHIFT\begin{proof}[#1]}{\end{proof}}

% \makeatletter
% \newcommand{\caseitem}{\@sitem}
% \newcommand{\@sitem}{%
%   \refstepcounter{\@enumctr}%
%   \item[\textbullet~\csname label\@enumctr\endcsname]}
% \makeatother

\newcommand{\step}[1][]{\item\maybeLabel{#1}}

\newenvironment{block}[2][]
{\step[#1]  #2 \begin{\STEPS}}{\end{\STEPS}}

\newenvironment{case}[2][]
{\step[#1] (case \currentlabel) \emph{#2}\begin{\STEPS}}{\end{\STEPS}}

\newcommand{\setHeader}{\markboth
{\footnotesize CS 145 turn-in \turnIn, \today, by \author, \SID}
{\footnotesize CS 145 turn-in \turnIn, \today, by \author, \SID}}

\newcommand{\setTurnIn}[1]{\newcommand{\turnIn}{#1}}
\newcommand{\setAuthor}[1]{\renewcommand{\author}{#1}}
\newcommand{\setSID}[1]{\newcommand{\SID}{#1}}

\pagestyle{myheadings}
\addtolength{\headsep}{0.2in}
\addtolength{\topmargin}{-0.4in}
\addtolength{\textheight}{0.4in}

%%% Local Variables:
%%% mode: latex
%%% TeX-master: "assignment-template"
%%% End:


\setTurnIn{???}
\setAuthor{Stanley Cohen (scohe001)}
\setSID{861114309}
\setHeader

\begin{document}

\begin{problems}

%%%%%%%%%%%%%%%%%%%%%%%%%%%%%%%%%%%%%%%%%%%%%%%%%%%%%%%% 1
 
\problem
 
\begin{theorem}[Problem 1]
 (Handshakes after the party) $N + 1$ couples attend a dinner. At the end of the dinner, 
  oen person (Alice) asks each other person to write down the number of distinct people
  with whom he or she shook hands at the dinner. Surprisingly, all numbers, $0, 1, 2,..., 2N$
  are written down. Assuming that no person shook hands with their partner, how many people 
  did Alice shake hands with at the dinner?
\end{theorem}
\begin{longFormProof}
 \step Assume we have a party of $N + 1$ couples where indiduals have 
        shaken hands with others but not with their own partner.
 \step Call the individuals of one such couple Alice and Bob.
 \step Assume that besides Alice, the number of hands shaken by each indivual ($h(i)$ for some individual $i$)
        is a unique integer in the range $[0,2N]$.
 \step By 3, there must exist an indidual who's shaken hands with $2N$ other individuals.
 \begin{block}[blockA]
  {Assume for contradiction that Bob is the person who's shaken hands with $2N$ individuals}
    \step Besides Alice and Bob, there are $2N$ guests.
    \step By 1, Bob cannot shake hands with Alice.
    \step By 5.1 and 5.2, Bob must've shaken hands with everyone but Alice.
    \step Handshaking is associative, so everyone but Alice has shaken hands with Bob.
    \step Contradiction by 3, someone (besides Alice) must've shaken hands with no one, but everyone shook hands with Bob.
 \end{block}

 \step By block~\ref{blockA}, Bob did not shake hands with $2N$ people.

 \smallskip
 \hrulefill
 \smallskip
 
       \begin{lemma}
         Given that a couple, $x,y$ where neither $x$ nor $y$ are Alice and $h(x)=2N$, it is implied $h(y)=0$.
       \end{lemma}
 
       \begin{longFormProof}[Proof of lemma.]
         \begin{block}[blockB]
            {Assume for contradiction that $h(y) \neq 0$}
         \step since $h(x)=2N$, $x$ cannot shake hands with $y$ and there are $2N$ people other 
              than $x$ and $y$ at the party, we know $x$ must've shaken hands with everyone but $y$.
         \step Since hand shaking is associative, everyone but $y$ has shaken hands with $x$.
         \step $h(y) \neq 0$ by a, but no one else has shaken hands with 0 people either, 
              which is a contradiction by 3.\
         \end{block}
         \step Therefore, if $h(x)=2N$, $h(y)$ must be $0$.
       \end{longFormProof}
 
 \vspace*{-1em}
 \hrulefill
 \smallskip

 \step By 3 and Lemma 1, there is some couple $x,y$ where $h(x)=2N, h(y)=0$.

 \step By removing the $x,y$ couple, we are left with couples whose handshake numbers range in: $[1, 2N-1]$ 
 \step If we remove this couple from the party, everyone will have shaken hands with one less person, since $x$ has shaken hands with everyone but $y$.
 \step After all the handshake numbers are decremented we are left with people in the range $[0, 2N-2]$ or $[0, 2(N-1)]$ which is a problem that is smaller than the previous problem by one, but still conforms to 1-3.
 \step By this process, we reduced the number of hands Alice had shaken by 1.
 \step We can repeat this process $N$ times before the only couple left is Alice and Bob.
 \step Thus, Alice shook hands with $N$ people, since we could reduce the problem $N$ times and each time would reduce the number of hands Alice had shaken by 1.
 
\end{longFormProof}
 
\end{problems}
\end{document}

%%% Local Variables:
%%% mode: latex
%%% TeX-master: t
%%% End:
