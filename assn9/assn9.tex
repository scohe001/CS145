\documentclass[11pt]{article}

\input{assignment-macros-2}

% TODO: CHANGE NAME AND SID TO YOURS:

\setTurnIn{9}  

\setAuthor{Stanley Cohen (scohe001)}
\setSID{861114309}
\setHeader

\begin{document}

\begin{problems}

  %%%%%%%%%%%%%%%%%%%%%%%%%%%%%%%%%%%%%%%%%%%%%%%%%%%%%%%%%%%% 
  %%%%%%%%%%%%%%%%%%%%%%%%%%%%%%%%%%%%%%%%%%%%%%%%%%%%%%%%%%%% 

  \problem % problem 1

  \begin{lemma}
    For any boy-optimal stable matching $M$ for any instance $I$,  $M$ is girl-pessimal.
  \end{lemma}
  \begin{longFormProof}
    \begin{block}[A]
      {Consider any instance $I$ and boy-optimal stable matching $M$.}

      \begin{block}[A1]
        {Suppose for contradiction that $M$ is not girl-pessimal.}

        \step Let $(b,g)\in M$ be a matched couple where $b$ is not $g$'s least-preferred possible partner. 

        \step Let $b'$ be the least-preferred possible partner of $g$.

        \step Let $M'$ be a stable matching in which $(b',g)$ are a couple.

        \step $b$ must prefer $g$ to his partner in $M'$, since $g$ is his best possible partner.

        \step Likewise, since $b'$ is $g$'s least preferred possible partner, $g$ prefers $b$ to $b'$.

        \step Thus, $(b, g)$ is an unstable pair in $M'$.

        \step Contradiction by 1.1.3., $M'$ is not a stable matching.
      \end{block}

      \step By block~\ref{A1}, $M$ is girl-pessimal. 
    \end{block}

    \step By block~\ref{A}, for any boy-optimal stable matching $M$ for any instance $I$,
    $M$ is girl-pessimal. 

  \end{longFormProof}

  \begin{shortFormProof}
    Assume for contradiction that we have some stable boy optimal matching $M$ on some instance 
    that is not girl pessimal. That is, there is some $(b, g)\in M$ where $g$ is paired with 
    a less preferred boy, $b'$ in some other stable matching, $M'$. However, $g$ is $b$'s most 
    preferred possible choice, and $g$ prefers $b$ to $b'$, so $(b, g)$ is an unstable pair in $M'$. 
    Therefore $M'$ cannot be stable, contradiction. Thus any boy optimal matching $M$ is also girl pessimal.
  \end{shortFormProof}

  %%%%%%%%%%%%%%%%%%%%%%%%%%%%%%%%%%%%%%%%%%%%%%%%%%%%%%%%%%%% 
  %%%%%%%%%%%%%%%%%%%%%%%%%%%%%%%%%%%%%%%%%%%%%%%%%%%%%%%%%%%% 

  \newpage 
  \problem  % PROBLEM 2

  \begin{enumerate}
  \item[(a)] First problem.
    
  \item[(b)] Original: 


    \begin{align*}
    \text{maximize } &x_1 + 2x_2\\
    \text{subject to } &2x_1 + x_2 \leq 10\\
    & x_1 + 3x_2 \leq 9\\
    & x_1, x_2 \geq 0.
    \end{align*}


    Dual:

    \begin{align*}
    \text{minimize } &10a + 9b\\
    \text{subject to } &2a + b \geq 1\\
    & a + 3b \geq 2\\
    &a, b \geq 0.
    \end{align*}

    \smallskip
    
  \item[(c)] % What is the optimal solution to the dual linear program?
    The optimal solution is $(a,b)=(1/5, 3/5)$, with value $37/5$. (found graphically).
  \end{enumerate}

  %%%%%%%%%%%%%%%%%%%%%%%%%%%%%%%%%%%%%%%%%%%%%%%%%%%%%%%%%%%% 
  %%%%%%%%%%%%%%%%%%%%%%%%%%%%%%%%%%%%%%%%%%%%%%%%%%%%%%%%%%%% 

  \medskip
  
  \problem  % PROBLEM 3

  \begin{enumerate}
  \item[(a)] Second Problem
    
  \item[(b)] Original: 


    \begin{align*}
    \text{minimize } &x_1 + 3x_2 + 5x_3\\
    \text{subject to } &x_1 + x_2 + x_3 \geq 1\\
    & x_1, x_2, x_3 \geq 0.
    \end{align*}


    Dual:

    \begin{align*}
    \text{maximize } &a\\
    \text{subject to } &a \leq 1\\
    & a \leq 3\\
    &a \leq 5\\
    &a \geq 0.
    \end{align*}

    These can be rewritten as simply:

    \begin{align*}
    \text{maximize } &a\\
    \text{subject to } &0 \leq a \leq 1\\
    \end{align*}

    \smallskip

  \item[(c)] Especially here in the dual, it becomes blindingly obvious that our optimal solution will be $a=1$ with a value of $1$.
  \end{enumerate}

\end{problems}

\end{document}
%%% Local Variables:
%%% mode: latex
%%% TeX-master: t
%%% End:
