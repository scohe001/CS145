\documentclass[10pt]{article}

\input{assignment-macros-2}

% TODO: CHANGE NAME AND SID TO YOURS:

\setTurnIn{14}  

\setAuthor{Stanley Cohen (scohe001)}
\setSID{861114309}
\setHeader

\begin{document}

\emph{The problems for this assignment will use the following graph, $G$}

\includegraphics[scale=0.5]{graph}

\begin{problems}

  \problem

  Here is an optimal solution vector, $X$ for the primal LP for shortest paths from $A$ to $E$ in $G$:

  $(x_{AB}, x_{AC}, x_{BC}, x_{CB}, x_{BD}, x_{BE}, x_{CD}, x_{CE}, x_{ED}) = (0,1,0,1,0,1,0,0,0)$ Yielding an optimal value of 6.

  \medskip 
  \problem
  Here is an optimal solution vector, $\pi$, for the dual LP for shortest paths from $A$ to $E$ in $G$:

  $(\pi_A, \pi_B, \pi_C, \pi_D, \pi_E) = (0, 3, 2, 5, 6)$ Yielding an optimal value of 6.
  
  \medskip 
  \problem

\begin{lemma}
  The dual solution $\pi$ above is optimal.
\end{lemma}

\emph{I'm doing the first two for turn-in 14, but what's to stop me from just doing this for turn-in 15? 
      I'm not sure what it is you're looking for here.}

\begin{longFormProof}

      \step We know from solving the primal that the optimal value for the length of the shortest path from $A$ to $E$ in $G$ is 6.

      \step By the definition of strong duality, since we found a solution to the dual with the same optimal value, this must be the optimal solution.

      \step QED.

\end{longFormProof}

\end{problems}

\end{document}

%%% Local Variables:
%%% mode: latex
%%% TeX-master: t
%%% End:
