\documentclass[11pt]{article}

\input{assignment-macros-0}

\setTurnIn{4}
\setAuthor{Stanley Cohen (scohe001)}
\setSID{861114309}
\setHeader

\begin{document}

\begin{problems}

%%%%%%%%%%%%%%%%%%%%%%%%%%%%%%%%%%%%%%%%%%%%%%%%%%%%%%%% 1

\problem

\begin{theorem}[Checkerboard]
  John has an 8x8 checkerboard and thirty-one 1x2 dominoes. John places the dominoes one by one 
    on the checkerboard. He places each domino to cover exactly two adjacent, previously uncovered 
    squares (two of the 64). When he is done, he has covered all but two of the squares. Is it 
    possible that the two squares remaining uncovered are the two white corner squares?
\end{theorem}
\begin{longFormProof}
  \begin{block}[blockA]
    {Consider any state of the board after some number of domino placements.}
    
    \step We will show that John maintains the following invariant at all times:\\
    \textit{The number of black squares covered is the same as the number of white squared covered.}

    \step The invariant is initially true, when no squares are covered.

    \begin{block}[blockB] 
    {Consider any way that John can place a single domino.}

      \begin{block}[blockC]
      {Assume the invariant holds just before John places the domino.}

        \step That is, just before the action, the number of uncovered white squares is the same as uncovered black squares.
        \step Each possible action covers two adjacent squares--one black square and one white square.
        \step Hence, the number of uncovered white and black squares always changes by (-1, -1) respectively.
        \step Hence, their values remain equivalent and the invariant holds, after the action.

      \end{block}

      \step By block 1.3.1, if the invariant holds before the action, it holds after.

    \end{block}

    \step By block 1.3, for each action, if the invariant holds before the action, it holds after.
    \step Since the invariant is initially true, and each of John's actions preserves it, the invariant holds.
    \step The invariant also implies that the last two uncovered squares can never both be the same color.
    \step It follows that the last two uncovered squares cannot be the two corner white squares.

  \end{block}

  \step By block~\ref{blockA}, after any sequence of actions that John might have taken, it is impossible for him to end with only the two white corners uncovered.
\end{longFormProof}


\end{problems}
\end{document}

%%% Local Variables:
%%% mode: latex
%%% TeX-master: t
%%% End:
