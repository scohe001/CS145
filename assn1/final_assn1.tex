\documentclass[11pt]{article}

\input{assignment-macros-0}
\usepackage{amssymb} %for empty set character

\setTurnIn{???}
\setAuthor{Stanley Cohen (scohe001)}
\setSID{861114309}
\setHeader

\begin{document}

\begin{problems}

%%%%%%%%%%%%%%%%%%%%%%%%%%%%%%%%%%%%%%%%%%%%%%%%%%%%%%%% 1
 
\problem
 
\begin{theorem}[Problem 1]
 If $x = \log_9(12)$, then $x$ is irrational.
\end{theorem}
\begin{longFormProof}
 \step By the definition of logs, we can say $x = \log_9(4*3) = \log_9(3) +
         \log_9(4) = \frac{1}{2} + \log_9(4)$
 
 \smallskip
 \hrulefill
 \smallskip
 
       \begin{lemma}
         Any rational plus an irrational will be an irrational number.
       \end{lemma}
 
       \begin{longFormProof}[Proof of lemma.]
         \begin{block}[blockL]
            {Assume for contradiction rational + irrational = rational}
         \step We can write the first rational as $a/b$ and the second as $c/d$
            for some $a, b, c, d \in \mathbb{Z}$ so $a/b +$ irrational $=c/d$.
         \step Subtracting $a/b$ from each side yields irrational $=c/d - a/b$
         \step $c/d - a/b$ can be represented as the fraction $\frac{ad-bc}{bd}$, which is rational, contradicting a.
         \end{block}
         \step Therefore, a rational + an irrational = an irrational.
       \end{longFormProof}
 
 \vspace*{-1em}
 \hrulefill
 \smallskip
 
 \step For $x$ to be rational, $y = \log_9(4)$ must be rational by Lemma 1.
 \step $y = \log_9(4)$ can be rewritten as $9^y=4$.
 \step Taking the square root of each side yields $3^y=2$.
 \begin{block}[blockA]
   {Assume for contradiction $y$ is rational, $y=p/q$ for some $p, q \in \mathbb{Z}$}
   
   \step we can rewrite $9^y=4$ as $3^{p/q}=2$ or $3^p=2^q$.
 
   \step $3^p$ will always be odd, while $2^p$ will always be even.
        Therefore, there can't exist any $p, q \in \mathbb{Z}$ to satisfy this equation.
 
   \step By step 3.5, $p$ and $q$ cannot be integers. Therefore y is not rational by contradiction.
 \end{block}
 
 \step By block~\ref{blockA}, and Lemma 1, $x$ cannot be rational
    since it is made up of the sum of an irrational number, y.
\end{longFormProof}
 
\end{problems}
\end{document}

%%% Local Variables:
%%% mode: latex
%%% TeX-master: t
%%% End:
